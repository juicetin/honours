\pagebreak
\section{Biodiversity}

Another beneficial aspect of Dirichlet Multinomial Regression it inherently provides information about the distribution of different habitats in a given region, allowing observations on biodiversity to be made without requiring extra post-processing steps such as clustering, which can be prohibitively expensive on datasets with millions of datapoints and tens (or more) of dimensions. Locate certain co-existence of species would involve searching over the space of predictions for the desired distributions of habitats. An alternative route to take, if no specific cohabitation in particular is being searched for, would be to isolate areas with the lowest entropy as explained in the previous section, then observe any evidence of biodiversity at these locations containing predictions that the model is confident about.

In contrast to Gaussian processes where the uncertainty  was generally high when there were mixes of labels in a particular area \todo{(don't know this yet! waiting for predictions to run)}, it is expected that the DM would be comparitively more confident that an even mix of labels exist in these areas. To obtain a sufficiently large area/number of points where two of the simplified four labels had a fairly even occurrence rate (with the other two labels only having trace amounts, if at all), sets of labels were repeatedly sampled with the variable condition that both their distributions lie within a certain range (for example, $[0.4, 0.5]$, or $[0.2, 0.3]$), until a segments were found that contained non-trivial amounts of at least two labels in the same area.

Because the 4-label case already aggregated similar classes from the original $24$ down to $4$, there was limited biodiversity to observe over the query space, requiring us to perform predictions over the full $24$ labels to be able to find more abundant occurrences of biodiversity. To give a qualitative visual representation of the distribution of labels over the query space when using all $24$ labels, the predictive heatmaps for each of them is shown in \Cref{fig:dm24_0-7}, \Cref{fig:dm24_8-15}, and \Cref{fig:dm24_16-23} below.

\begin{figure}[H]
    \includegraphics[clip, width=\columnwidth]{dm24_heatmaps_0-3.pdf}\\
    \includegraphics[clip, width=\columnwidth]{dm24_heatmaps_4-7.pdf}
    \caption{\todo{(need to redo these to show more colour @\_@)}}
    \label{fig:dm24_0-7}
\end{figure}

\begin{figure}[H]
    \includegraphics[clip, width=\columnwidth]{dm24_heatmaps_8-11.pdf}\\
    \includegraphics[clip, width=\columnwidth]{dm24_heatmaps_12-15.pdf}
    \caption{\todo{(need to redo these to show more colour @\_@)}}
    \label{fig:dm24_8-15}
\end{figure}

\begin{figure}[H]
    \includegraphics[clip, width=\columnwidth]{dm24_heatmaps_16-19.pdf}\\
    \includegraphics[clip, width=\columnwidth]{dm24_heatmaps_20-23.pdf}
    \caption{\todo{(need to redo these to show more colour @\_@)}}
    \label{fig:dm24_16-23}
\end{figure}

Given the multi-label distribution predictions, there are any number of ways to either quantatively or qualitatively measure biodiversity. To attempt a combination of the two, experiments were run to identify areas where $2, 3, 4,...$ etc. labels co-occurred. Depending on the context in which the data is used, the definition of co-occurence is also entirely flexible. The formulation taken here is that for any given point (and by extension, region, where this pattern occurs frequently enough that the number of points fitting the criteria are dense enough to described as a cluster), $n$ number of labels are said to co-occur if \todo{need a definition here}

areas where $2$ or more labels co-existed were searched, but where their occurrence ratios lie within a certain range. The examples we show below limited this range to $0.03$ - however, this is arbitrary, and also completely tunable depending on the specifics of the benthic region, as well as the goals of obtaining biodiversity metrics. These searches were automated over the query space to find sections where the average variance was noticably lower than other parts of Scott Reef, indicating that the model had a higher confidence of a particular mix of habitats in the given section.

% \todo{[PLACEHOLDER]}
% We can visually observe that labels $7, 8, 10$ (variants of sand) occur in the same regions, with the following occupancy rates, beyond areas containing only negliglble traces ($\equiv0.00-0.02$) $7$ at a $\approx0.12$ occpuance rate, $8$ at $\approx0.18-0.20$, and $10$ at $\approx.17$.

What becomes apparent is that in the areas where the DM is confident of a mix of certain set of predominant labels, the GP is instead equally uncertain of each of them with a considerably higher variance, which is misleading information when taken at face value. For example, this sort of uncertainty may be taken into consideration purposes, where autonomous vehicles are used to collect data, or in making decisions with regards to conservation efforts. In the first scenario, resources are being wasted on areas where models such as the DM can be confident of a particular distribution of labels, whereas in the second, important conservation actions may be withheld if the \textit{certainty} of information is brought into question. For example, in an area that contains a particular mix of coral and bleached coral, a DM has the potential to make a confident prediction of their coexistence, whereas a GP would make predictions where their respective probabilities in a one-vs-all classifier may be close to their distribution in the area, but have a high noise factor.


