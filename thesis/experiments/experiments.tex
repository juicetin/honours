\chapter{Experiments and Results} \label{chap:experiments}

To show that using Dirichlet Multinomial Regression provides richer and more valuable information than single-output or deterministic methods alone, we ran experiments on the data obtained from the ACFR's Sirius AUV and Schmidt's Falkor. In this chapter, we first assess the performance and usefulness of information of single output labels, to first highlight the need for models that can effectively perform multi-output predictions.

As seen in the previous chapter, a key benefit to applying the Dirichlet Multionomial distribution to the data is that it able to naturally perform multi-output predictions on label distributions that correctly sum back to $1$. To illustrate this point, SVMs, Linear Regression, K-Nearest Neighbour, Random Forest, and Gaussian Process Regression were all coerced to perform multiple predictions across each label's normalised distribution values, where the results were compared with those of a Dirichlet Multinomial Regressor. However, an important point to keep in mind is that these models do not maintain the constraint of predictions per point summing back to $1$, but they have been included in the experiments to illustrate if they can still provide reasonable results despite being an inherently `incorrect' model, with the advantage of having implementations more readily available in open source libraries, and being more exhaustively studied in literature in general.

We also explored how to use the data to extract information about biodiversity and the corresponding confidence, indicated by the predictive variance in the case of Gaussian Processes and Dirichlet Multinomials. Moreover, to contrast the Dirichlet Multinomial's ability to naturally provide information about co-existing habitats with certainty, we compared the regions in which the Dirichlet Multinomial was certain with those in Gaussian Process predictions, looking at both the overlapping areas and the corresponding level of variance observed in both models.

\section{Training Data}
To perform our experiments, bathymetry data and images of Scott Reef Central were used (\Cref{fig:scottreefaerial}). The bathymetry data was collected using Eric Schmidt's Falkor a ship dedicated to marine research, with the depth for a large portion of the reef collected . Over 700GB of `truthing' image data was collected by The University of Sydny's Australian Centre for Field Robotic's Sirius autonomous underwater vehicle (AUV). The training set provided already had labels assigned, which was a result of previous efforts by Daniel Steinberg using Variational Dirichlet Processes that performed the unsupervised clustering \citep{steinberg11}. 

On close inspection, the UTM\footnote{UTM is short for Universal Transverse Mercator, a coordinate system that splits the Earth up into a grid-like structure, where a location is given by the grid key, then an $(x,y)$ coordinate in metres that defined a position from the `origin' of the given grid.} coordinates in the training set do not correspond to the original data available from \cite{squidle} - this was because the exact point of retrieval for the bathymetry and image were not exact matches. To account for this, labels corresponding to bathymetry points were in fact taken from the closest images, rather than exact longitude/latitude or UTM matches, although the UTM coordinates in the training data itself remains as the original. Due to data wrangling previously done by Asher Bender, features of rugosity (roughness) and slope are also available, each measured at resolutions of $2$m, $4$m, $8$m, and $16$m.

\begin{figure}
    \includegraphics{scott_reef_aerial_cropped.jpg}
    \caption{Aerial shot of Scott Reef from \cite{NASA:SRI}}
    \label{fig:scottreefaerial}
\end{figure}

\section{Data Preprocessing}

\subsection{Downsampling the Data}
As the purpose of using Dirichlet Multinomial Regression was to be able to model the distribution of habitat label occurrences over an area, we downsampled the combined 2011, 2015 dataset which was at a significantly higher resolution than the 2009 dataset \todo{(calculate how much perhaps)}. To do this, a `grid' of squares was essentially mapped onto original space, and points in each grid grouped together into a single point. For the multi-output labels, all label counts in each lower-resolution grid were summed, whereas for the single label case, the label was selected randomly \textit{but} with more weighting to the non-sand labels, to slightly increase the variance of labels observed. The reason for doing this was because of the dominance of the 'sand` habitat - random selection when only one absolute label per point was present resulted in an even larger portion of sand than in the original dataset, causing predictions on the full query data to sometimes result in almost $100\%$ sand, even moreso than was already the case in \autoref{fig:det4maps}.

% Two methods of downsampling in particular were tested. The first coarser approach involved simply taking the space in which the data was collected and placing grids of fixed size over them as in \cref{fig:gridsplit}, binning all points falling within each grid into a single datapoint. Each of these data points contained multiple points from the original dataset with their own counts for each of the possible labels, so the downsampled points simply took the sum of all the label counts in each fixed grid. 
% 
% The second summed label counts in the same way, but clusters were instead formed by first calculating the full dendrogram on the 16502 entries in the training data, and forming groups such that none had more than 5 of the original points within them, and the sub-clusters (at each level of the dendrogram) were no more than a 21 metres away from one another. As can be seen in \cref{fig:dendrogram}, the gradual merging into the single supercluster was quite consistent, indicating the original datapoints were mostly evenly distributed.
% 
% \todo{(cut down this section (probably down to one paragraph) - and also talk about how for the single label case, instead of summing counts for the new 'label', labels were chosen at random (for more variance in the GP which is otherwise too certain about everything being sand))}
% 
\begin{figure}[H]
    \includegraphics[scale=0.55]{training_map_fixedgrid.pdf}
    \caption{Fixed-sized grids placed over training data \todo{(redo this plot nicely!)}}
    \label{fig:gridsplit}
\end{figure} 

% \begin{figure}[H]
%     \centering
%     \includegraphics[scale=0.6]{dendrogram.pdf}
%     \caption{Dendrogram of training data}
%     \label{fig:dendrogram}
% \end{figure}

\subsection{Simplifying labels}
Another preprocessing step that was performed was the aggregation of habitat labels. The original training data contained 24 separate labels determined through an automated clustering procedure using Dirichlet Processes. Because of the uneven distribution of these labels (\autoref{fig:singlelabeldistr} and \autoref{fig:multilabeldistr}), with the occurrence of some too insignificant for any machine learning algorithms to pick up, they were simplified in collaboration with ecological experts, who manually identified which of the 24 labels were in fact of the same class - for example, 5 separate classes of coral may have been indistinguishable to the average person, and were hence grouped into a single label. This allowed the near-non-occurring labels to be grouped together with more commonly occurring ones, whilst also allowing a different level of granularity in training models/forming predictions that could be used if only an approximation equivalent to observable human differences of an area's benthic map were required. Moreover, due to the unsupervised nature of the labeling, certain clusters were notably \textit{inconsistent} with the rest, for example when sea cucumbers became the identifying feature of one of the 24 labels.

\begin{table}[H]
    \centering
    \begin{tabular}{|c| c|}
        \hline
        simplified & original \\\hline
        0 & 1, 2, 18, 20, 21, 23, 24 \\
        1 & 3, 5, 10, 16, 17, 19, 22\\
        2 & 13, 14, 15 \\
        3 - Sand & 4, 6, 7, 8, 9, 11, 12 \\
        \hline
    \end{tabular}
    \caption{Full-simplified label mappings \tiny{\todo{label mappings - sand, coral, patchy coral, (?) halameda, rhodliths}}}
    \label{table:labelmappings}
\end{table}

\begin{figure}
    \includegraphics[scale=0.17]{class_mosaic_24_classes.jpg}
    \caption{Samples of images from each of the full 24 classes}
    \label{fig:24classes}
\end{figure}

\begin{figure}[H]
    \begin{minipage}{.47\linewidth}
        \includegraphics[width=\linewidth]{hist_full_labels.pdf}
        \caption{Distribution of labels in original dataset}
        \label{fig:singlelabeldistr}
    \end{minipage}
    \hfill
    \begin{minipage}{.47\linewidth}
        \includegraphics[width=\linewidth]{hist_full_multi_labels.pdf}
        \caption{Distribution of labels in multi-label outputs}
        \label{fig:multilabeldistr}
    \end{minipage}
\end{figure}

\begin{figure}[H]
    \begin{minipage}{.47\linewidth}
        \includegraphics[width=\linewidth]{hist_simple_labels.pdf}
        \caption{Distribution of simplified labels in original dataset}
        \label{fig:singlelabeldistr}
    \end{minipage}
    \hfill
    \begin{minipage}{.47\linewidth}
        \includegraphics[width=\linewidth]{hist_simple_multi_labels.pdf}
        \caption{Distribution of simplified labels in multi-label outputs}
        \label{fig:multilabeldistr}
    \end{minipage}
\end{figure}

Note that from this point onwards, we will be working with the reduced feature set, in line with the aim of the paper to show the advantages of dirichlet multinomial regression when studies (environmental or otherwise) are limited to lower resolution data where strictly assigning only a single label to the features at a given data point is not representative of the otherwise rich information available. This restriction is a realistic one, because to be able to monitor large portions of the ocean for conservational and management reasons amongst others, data needs to be collected economically en-masse - and this means not collecting very high resolution data that would attract large costs at scale.

\subsection{Coordinates as features}
Due to the abundant bathymetry data that was available in the form of depth, rugosity and aspect at each available data point, the coordinates themselves were not included in the feature space for experiments. Whilst it is logical that in a natural environment, areas that were spatially close would also have similar properties, this should not be relied upon, and other intrinsic properties should be the basis upon which predictions are made. Forming predictions on the full query dataset using a random forest supports this notion quite strongly - whilst 10-fold cross validation using the coordinates as features had a notably higher F-score of 0.61 compared to 0.40 without, the unnaturally straight split between the left and right segments (\autoref{fig:rf_w_coords_preds}) over a 12km region suggests that the predictive map is flawed. Moreover, by including the coordinates as a training feature, an assertion is made about the direct relationship between a benthic location and the habitat class/es it contains, despite its other physical properties such as depth, aspect, etc - an assumption that should not be embedded into the model before fitting it to the data even begins.

\begin{figure}[H]
    \begin{minipage}{.49\linewidth}
        \includegraphics[width=\linewidth]{full_predictions_randomforest.pdf}
        \caption{Full predictive map using Random Forests including coordinates as features}
        \label{fig:rf_w_coords_preds}
    \end{minipage}
    \hfill
    \begin{minipage}{.49\linewidth}
        \includegraphics[width=\linewidth]{full_predictions_randomforest_nocoords.pdf}
        \caption{Full predictive map using Random Forests excluding coordinates as features}
        \label{fig:rf_wo_coords_preds}
    \end{minipage}
\end{figure}

\subsection{Preprocessing and Feature Projection}
To maximise performance of the algorithms used across the experiments, a number of preprocessing steps were taken to improve the predictions made. The features in the data were first scaled, where each feature was centred to the mean with unit variance), then normalised over each future such that they had unit length \todo{(include plots of the diff approaches across DM/GP/others, ref plots)}. To allow the algorithms tested to learn the data and its complexities better, projecting the data into higher dimensional space was required. Full quadratic projections ($x_0, x_1, x_2 \Rightarrow x_0^2 ,x_1^2 ,x_2 ,2x_0x_1 ,2x_0x_2 ,2x_1x_2$) and squared terms with a $1$ bias terms ($x_0, x_1, x_2 \Rightarrow x_0 + x_1 , x_2 ,x_0^2 ,x_1^2, x_2^2, 1$) were both tested. The latter was chosen, one of the reasons being that it resulted in an lower average error when performing 10-fold cross-validation using Dirichlet Multinomial Regression (\autoref{table:dmbasicresults}). It was also chosen to allow predictions to run faster, and for the Markov Chain Monte Carlo for Dirichlet Multinomial Regression later in this chapter, \textbf{significantly} reduce the number of dimensions that need to be traversed - considering the number of weights required is features $\times$ number of labels, this would correspond to the 4-label data needing $19*4=76$ vs $55*4=220$ weights, and the full 24-label case requiring $19*24=456$ vs $55*24=1320$ weights.

\todo{(ref plots)}

\todo{(show plots)}

% The results from the Experiments detailed in Chapter 4 are listed below. The range of possible class values in some cases have been stretched beyond the existing class labels so that values align between different outputs to allow for easy, direct visual comparison \todo{(this hasn't actually happened yet)}. Note that the results to the above experiments will include those of both non-downsampled and downsampled results, as well as the full set of 24 labels as well as simplified ones.
% 
% Due to the low occurrence of some labels in the original dataset though, they have ended up being ommitted in predictions - these are excluded from the colour schemes of the benthic maps generated, so that those that do occur can be given more distinct colours from one another as to better differentiate between the habitats of a map, as well as allow a consistent comparison of across different maps. \todo{(again, hasn't happened yet)}

% The associated generated maps from the experiments are also provided here, but proper evaluation of them, such as what habitat clusters and relationships can be gleaned, will be explored in Chapter 5. 


\section{Deterministic Approaches (Single Output)} \label{chapsec:deterministic}

We first briefly review the machine learning techniques more commonly used in benthic habitat mapping first, to get an idea for the sort of maps generated as well as their performance for the given dataset. To quantifiably compare their predictions, we calculate their unweighted f-scores. The \textit{f-score} of predictions are a measure of accuracy in classification problems that takes into account both precision and recall across each possible label, and is calculated by $2*\frac{\text{precision*recall}}{precision+recall}$. The use of unweighted f-scores means, we calculated the \textit{f-score} separately for each label in the predictions, and simply took the average of them. This was chosen in favour of weighted f-scores that provide a larger weight for more frequent labels as the high occurrence of sand would hide the fact that the other labels are constantly incorrectly predicted, if this was the case. 
% Logistic regression has been included here despite containing 'probabilistic` predictions in the form of regression values passed through the \textit{logistic} function, and as such the results displayed are a result of simply taking the argmax over possible the predictive probability over possible for each datapoint. Those probabilistic outputs are useful for comparisons with those of Gaussian Processes, however, which will be explored in the next section. \todo{(this GP-LR comparison hasn't happened yet)}

\begin{table}
    \centering
\begin{tabular}{|c|c|c|c|c|c|}
    \hline
    Algorithm & 10F-CV F-score & 10F-CV Accuracy & Label type\\\hline
    SVC & 0.21514 & 0.75554 & 4 labels \\
    LogisticRegression & 0.33713 & 0.77001 & 4 labels \\
    KNeighborsClassifier & 0.4714 & 0.7796 & 4 labels \\
    RandomForestClassifier & 0.4737 & 0.79406 & 4 labels \\
    SVC & 0.10355 & 0.29408 & 24 labels \\
    LogisticRegression & 0.13335 & 0.31389 & 24 labels \\
    KNeighborsClassifier & 0.22593 & 0.33093 & 24 labels \\
    RandomForestClassifier & 0.22015 & 0.3405 & 24 labels \\
    \hline
\end{tabular}
\label{table:detresults}
    \caption{Performance of common machine learning models}
\end{table}

While the accuracy of the Logistic Regressor, kNN, and Random Forest Classifier are reasonable (above $0.75$), the former two's f1-scores are very poor at $0.33$, with the latter two at just below $0.5$, which is an equally undesirable result. Looking at the ratio of available labels in the downsampled data in the 4-label case ($232,  470,  446, 3548$ for labels 0, 1, 2, 3 respectively) reveals that label 3 accounts for $0.7556$ of the dataset - a value very close to the accuracy of predict. The weighted f1-score of a `naive' classifier that always predicts label 3 has an accuracy of  $0.75554$ and an average f-score of $0.215$ - highlighting the fact that these simpler models are not able to produce results that confidently outperform simply guessing one label for any given datapoint. \Cref{fig:det4maps} visualises the predictions from \Cref{table:detresults} on the full query data for the 4 and 24-label data respectively. The Support Vector Machines (SVM) that generally provides moderately respectable real-world performance has noticably failed to predict anything other than sand throughout the query space, hinting the underlying data has complexities that require more complex models to explain it. The predictive maps generate using Logistic Regression and kNN bear noticable similarities in many areas of the map, while Random Forests identified regions that the others didn't. 

\begin{figure}[H]
    \includegraphics[scale=0.61]{det4_preds.pdf}
    \includegraphics[scale=0.45]{det4_preds_colourbar.pdf}
    \caption{Full aggregated label predictive map using SVMs, Logistic Regression, kNN, and Random Forests}
    \label{fig:det4maps}
\end{figure}

\begin{figure}[H]
    \includegraphics[scale=0.61]{det24_preds.pdf}
    \includegraphics[scale=0.45]{det24_preds_colourbar.pdf}
    \caption{Full predictive map using SVMs, Logistic Regression, kNN, and Random Forests}
    \label{fig:det24maps}
\end{figure}

The maps in \Cref{fig:det4maps} (with the exception of the SVM-generated one) provide some insight into where certain habitats occur in Scott Reef. However, as the results of the other three models were comparable, particularly for Random Forests and K-Nearest Neighbours, it is not obvious which one is more `trustworthy', and what prediction to take in areas that they disagree on. One piece of information that can aid in this regard is if a level of \textit{confidence}, which we explore in the next section.


\section{Probabilistic Approaches (Single Output)}

In this section, we will add an extra layer of information to our models' outputs - the confidence of the label predictions made. When predictive variance at each point is given, a large variance would indicate a low level of confidence as the predicted value is any within a large range, whereas a small variance indicates a high level of confidence in a prediction, as the possible range of values is only a small one. For this, we need probabilistic models that naturally provide this desired variance in its predictions. In particular, as we saw in \Cref{chapsec:gpc}, Gaussian process classification is a good option for this.

\subsection{Gaussian Process Classification}

While f-scores and accuracy are still assessed via $10$-fold cross validation whilst using Gaussian process classification, we introduce another metric, area under the receiver operating curve, to make use of the fact that the one-vs-all Gaussian process classifier provides a \textit{likelihood} of each label's membership at each datapoint. This encapsulates that at any given point, predictions will (almost) never be 100\% certain - every single possible label, however unlikely, will have a probability assigned to it. 

Area Under The Receiver Operating Curve
\todo{TODO}

\todo{This round's acc: 0.2822085889570552, f1: [ 0.          0.46153846  0.29787234  0.47368421  0.          0.15384615
  0.34482759  0.05263158  0.22727273  0.4         0.28571429  0.28571429
    0.33333333], f1 avg: 0.25511038180964307, auroc: 0.41135738361342433, auroc avg: 0.41135738361342433}
\todo{This round's acc: 0.3936842105263158, f1: [ 0.          0.36734694  0.47787611  0.50943396  0.          0.30379747
  0.47727273  0.          0.27027027  0.44444444  0.14285714  0.375
    0.33333333  0.52173913  0.          0.85714286], f1 avg: 0.31753214883402126, auroc: 0.5498975366846471, auroc avg: 0.5498975366846471}
\begin{table}[H]
    \centering
\begin{tabular}{|c|c|c|c|c|}
    \hline
    AUROC & Accuracy & F-score & Labels used\\\hline
    0.85 & 0.83 & 0.53 & 4 labels \\
    0.55 & 0.39 & 0.32 & 24 labels \\
    \hline
\end{tabular}
    \caption{Gaussian proces 10-fold cross validation errors on full and simplified labels}
    \label{table:gperrs}
\end{table}

% \todo{May ditch this table}
% 
% \begin{tabular}{|c|c|c|c|c|c|c|c|}
%     \hline
%     No. points & Type of split & Type of GP & Number of runs & AUROC & Notes & F1-score \\\hline
%     500     & Even       & GP     &  10        & 0.86534    &                     &         \\
%     500     & Stratified & GP     &  10        & 0.80136    &                     &         \\
%     1000    & Even       & GP     &  1         & 0.87626    & Deterministic       & 0.56208 \\
%     1000    & Even       & PoEGP  &  5         & 0.80973    &                     & 0.47481 \\
%     1000    & Even       & PoEGP  &  200       & 0.80186    &                     & 0.47595 \\
%     1000    & Even       & GPoEGP &  5         & 0.80864    &                     & 0.51018 \\
%     1000    & Even       & GPoEGP &  200       & 0.80105    &                     & 0.47748 \\
%     1000    & Even       & BCM    &  5         & 0.80682    &                     & 0.48167 \\
%     1000    & Even       & BCM    &  200       & 0.80421    &                     & 0.48227 \\
%     1000    & Even       & GPy    &  1         & 0.87638    & RBF, EP (default)   & 0.57013 \\
%     \hline
% \end{tabular}
% \todo{(look at AUROC/AUC and log probabilities as well)}

\begin{figure}
    \includegraphics[scale=0.9]{gp4_allpreds.pdf}
    \caption{Gaussian process classification predictions over full query space for simplified labels}
    \label{gp4_allpreds}
\end{figure}

\todo{highlight areas with low/high certainty, etc. NOTE - investigate the areas with visually even splits of two labels - e.g. right-side arms of label 1,2, and smaller patches in the bottom left corner of label 0,3 - show that uncertainty about whether those areas are label 1 or 2, 0 or 3 respectively, is (should) be high, and that taking argmax for the sake of visual representation within a single image hides this information}

\todo{(talk about variance and extra probabilistic info to be gained)}

\todo{(maps of 4-label full predictions)}

\todo{(maps of all-label full predictions)}

\subsection{Ensemble Gaussian Process Approximations}

Although Gaussian processes provide the benefit of the possible variance for every prediction, the matrix inversion steps required and their $O(n^3)$ complexity prevents scaling of fitting the model beyond several thousand points (that even on high-end consumer hardware is impractically time-consuming), and predictions an order of magnitude above that. Using the naive Gaussian process, model fitting on the aggregated $4$-label data with $5000$ data points took over a day\todo{(double check this again!)}, with the predictions of the $500,000$ training points taking another day. It would be helpful to note that the formulation of Gaussian process classification used in this study (binary one-vs-all classifiers per label) means the complexity not only scales with number of points, but also in the number of possible labels, with each label requiring another underlying Gaussian process to account for it - for example, data containing $24$ labels would require 24 separate binary Gaussian processes. Due to hardware constraints, the $24$-label cases were not run to completion using the naive approach. As a point of reference, model fitting and predictions were timed for the basic Gaussian process, but only for a subset of the data, as shown below.

\begin{table}[H]
    \parbox{.85\linewidth}{
    \centering
    \begin{tabular}{|c|c|C{3cm}|}
        \hline
        Points & Labels & Time Taken (hh:mm:ss) \\\hline
        1000 & 4  &  00:00:37 \\
        2000 & 4  &  00:02:12 \\
        3000 & 4  &  00:11:53 \\
        4000 & 4  &  00:33:26 \\
        4700 & 4  &  01:15:41 \\
        1000 & 24  & 00:00:54 \\
        2000 & 24  & 00:07:03 \\
        3000 & 24  & 00:23:42 \\
        4000 & 24  & 00:39:43 \\
        4700 & 24  & 01:07:15 \\
        \hline
    \end{tabular}
    \label{table:gpensemble-results}
    \caption{Gaussian process model fitting runtimes, using gradually increasing number of points for both simplified and full-label cases}
}
\end{table}
\begin{table}[H]
    \parbox{.85\linewidth}{
    \centering
    \begin{tabular}{|C{2cm}|c|C{3cm}|}
        \hline
        Training points used & Labels & Time Taken (hh:mm:ss) \\\hline
        1000 &  4 &  00:01:23 \\ % (00:49:57??) 
        2000 &  4 &  00:04:18 \\
        3000 &  4 &  00:10:02 \\
        4000 &  4 &  00:17:35 \\
        4700 &  4 &  00:32:41 \\
        1000 & 24 &  \\
        2000 & 24 &  \\
        3000 & 24 &  \\
        4000 & 24 &  \\
        4700 & 24 &  \\
        \hline
    \end{tabular}
    \label{table:gpensemble-results}
    \caption{Gaussian process predictoin runtimes, using gradually increasing number of points for both simplified and full-label cases}
}
\end{table}

On the other hand, the benefits of Gaussian processes need not be sacrificed on account of this disadvantage of Gaussian processes, as approximations exist to break the original dataset into smaller chunks, allowing parallelisation and model fitting per smaller set of data that is only limited by the available hardware. As seen in \Cref{chapsec:gp-approx}, one such method is to use ensembles of Gaussian processes that allow trivial parallelisation. These experiments were carried out on \textbf{d2.8xlarge} Elastic Container (EC2) instances from Amazon, with the following specifications:
\begin{table}[H]
    \centering
    \begin{tabular}{|c|c|c|c|c|}
        \hline
        Instance Type & vCPUs* & Memory(GB) & Physical Processor & Clock Speed \\\hline
        d2.8xlarge & 36 & 244 & Intel Xeon E5-2676v3 & 2.4Ghz\\\hline
    \end{tabular}
    \label{table:ec2specs}
    \caption{Amazon EC2 Instance Machine Specifications}
\end{table}

To illustrate the usefulness of these approximation methods, experiments were run to measure their accuracy and f-scores, but also the time needed to run them particularly compared to naive Gaussian processes and usefulness of the maps - providing a point of reference when looking at Dirichlet multinomials in the next chapter. The abbreviations of the ensemble methods will be used in the following tables: product of experts (PoE), generalised product of experts (GPoE), Bayesian committee machines (BCM), robust Bayesian committee machines (rBCM).

\begin{table}[H]
    \parbox{.45\linewidth}{
        \centering
    \begin{tabular}{|C{2cm}|c|c|c|c|}
        \hline
        Ensemble method & F-score & Accuracy & AUROC \\\hline
        PoEGP & 0.35 & 0.76 & 0.71 \\
        GPoGPE & 0.33 & 0.75 & 0.69\\
        \hline
    \end{tabular}
    \caption{Gaussian process approximation results for simplified (4) labels}
}
    \hfill
    \parbox{.45\linewidth}{
        \centering
    \begin{tabular}{|C{2cm}|c|c|c|c|}
        \hline
        Ensemble method & F-score & Accuracy & AUROC \\\hline
        PoEGP & 0.17 & 0.18 & 0.34 \\
        GPoGPE & 0.19 & 0.21 & 0.34 \\
        \hline
    \end{tabular}
    \caption{Gaussian process approximation results for full 24 labels}
}
    \parbox{.45\linewidth}{
        \centering
    \begin{tabular}{|C{2cm}|C{2cm}|C{2cm}|}
        \hline
        Ensemble Type & Labels used & Time Taken (hh:mm:ss) \\\hline
        PoE   & 4   &  00:00:23 \\
        PoE   & 24 &  00:00:37 \\
        GPoE  & 4   &  00:00:23 \\
        GPoE  & 24 &  00:00:40 \\
        \hline
    \end{tabular}
    \label{table:gpensemble-training}
    \caption{Gaussian process ensemble training runtimes for all $5000$ Training points}
}
    \hfill
    \parbox{.45\linewidth}{
        \centering
    \begin{tabular}{|C{2cm}|C{2cm}|C{2cm}|}
        \hline
        Ensemble Type & Labels used & Time Taken (hh:mm:ss) \\\hline
        PoE   & 4   & 00:03:35 \\
        PoE   & 24  & 01:54:47 \\
        GPoE  & 4   & 00:03:31 \\
        GPoE  & 24  & 01:53:30 \\
        \hline
    \end{tabular}
    \label{table:gpensemble-predictions}
    \caption{Gaussian process ensemble prediction runtimes for all $500000$ test points}
}
\end{table}

\begin{figure}
    \includegraphics[scale=0.64]{gpogpe4_allpreds.pdf}
    \caption{Argmax map of generalised product of Gaussian process experts for simplified labels}
    \label{fig:gpogpe4}
\end{figure}

\begin{figure}
    \includegraphics[scale=0.7]{gpogpe23_allpreds.pdf}
    \caption{Argmax map of generalised product of Gaussian process experts for full 23 labels \todo{match colours better}}
    \label{fig:gpogpe23}
\end{figure}

Given that random forests were able to best fit the training data in \Cref{chapsec:deterministic}, it is promising that the the simplified generalised product of Gaussian process expert predictions, whereby the label with the maximum probability at each point is taken to be the `absolute' truth, bears close a resemblance to it. If Gaussian processes were used in practice, there is still the variance that can be used to make better use of the predictions. 

While the product of experts and its generalised counterpart appear to have the comparable performance measured on error alone, the former failed to predict labels $0$ through $2$ in the simple case, and the corresponding labels for the full $24$. The generalised product of experts however, performed quite well, identifying similar habitat regions to the Dirichlet multinomial, as shown in \Cref{chapsec:dm}. Even with these improvements to running time whilst retaining probabilistic information, there is still important functionality missing from these models - the counts of other labels at every point that was available in the original data. The methods explored thus far are not able to deal with such multi-output data, and other models must be used to do this. Moreover, the level of parallelisation used to speed up operations would only be available on server hardware that can become expensive if constantly relied upon. Although scaling Gaussian processes was possible here using hardware and approximation methods, sacrifices in correctness were made. 

To use in live scenarios, such as an AUV with previous bathymetry data collecting new images, the time it would take to run approximations before being able to calculate a desired path would render the vehicle inactive for most of the time. The several minutes achieved by the ensemble methods on the simplified data above was a result of parallelising over $30$ cores and being able to utilise over $200$GB - not the sort of hardware that would be appropriate for a small deep sea vessel. Not to mention, the sort of post-processing required in simplifying labels requires human intervention, and would not be something that can be automated on an expedition into previously unexplored areas. Hence, different methods would be needed that can perform similar tasks that can provide measures of uncertainty in realtime, all while being able to run quickly on hardware with limited capabilities.

\pagebreak

\section{Multi-Output Predictions} \label{chapsec:dm}

Looking at the deterministic maps from \Cref{fig:det4maps} as well as \Cref{fig:gpogpe4} and \Cref{fig:gpogpe23} using the simplified ensemble Gaussian process approximations, it can be observed where clusters of certain habitats are - but what can't be easily obtained, or at least automated without non-trivial extra effort, is identifying exactly \textit{where} these clusters are, and the frequency of co-occurrence between the different habitats. This is a consequence of only having a single label per point, but considering the area covered by a single data point, it is unrealistic to imagine the entire surface being only a \textit{single} label. Thus, we explore how to predict the \textbf{distribution} of labels at each point as represented in the original data, to provide richer habitat maps that naturally illustrate the co-occurrence of different habitats.

As a means of effectively visualising the separate labels, we need to look at the normalised distribution of habitat classes for each label separately. In the maps below created from each model's respective predictions, each class is represented on a separate heatmap, with the occurence (with a maximum of $1$, when an area is predicted to \textit{only} contain that label) indicated by the colour bars included above each map. This allows initial observations to be made of where certain labels are more abundant than others. This representation allows a user/viewer to easily manually identify where and which labels have a high occurrence (without being required to constantly check which specific colour a label was, etc.), but also larger areas where habitats co-occur.

% \subsection{Coercion of Common Regression Machine Learning Algorithms} \label{ss:commonMLcoercion}
% To do this, we first look at the regression version of algorithms used in the previous section, instead applied individually to each of the label distribution values in the original dataset, rather than simplifying them down to a single habitat label. 
% 
% \begin{table}[H]
%     \centering
%     \title{\large{\textbf{Multi-output average Errors}}}
%     \begin{tabular}{|c|C{2cm}|C{2cm}|C{2cm}|C{2cm}|C{2cm}|}
%         \hline
%         Algorithm & Average Error & Labels Used & Average Row Sum* & Min Row Sum & Max Row Sum\\\hline
%         SVR & 0.2073 & 4 & 0.7407 & 0.1687 & 1.4418\\
%         LinearRegression & 0.1827 & 4 & 0.5224 & 0.1136 & 0.8421\\
%         KNeighborsRegressor & 0.1698 & 4 & 0.5365 & 0.0 & 1.0\\
%         RandomForestRegressor & 0.1722 & 4 & 0.5267 & 0.0 & 1.4\\
%         SVR & 0.0983 & 24 & 1.8567 & 1.826 & 1.8834\\
%         LinearRegression & 0.0463 & 24 & 0.155 & -0.1624 & 0.4006\\
%         KNeighborsRegressor & 0.0434 & 24 & 0.1655 & 0.0 & 1.0\\
%         RandomForestRegressor & 0.0441 & 24 & 0.1842 & 0.0 & 1.3\\
%         \hline
%     \end{tabular}
%     \label{table:dmbasicresults}
%     \caption{Average errors of multi-output versions of single-output regression methods}
% \end{table}
% 
% Note that the smaller errors when predicting using all 24-labels do not mean that it is `better' to do so - because the values in the training set are smaller to begin with, it follows that the variance in predictions will also be smaller as a result. To visualise how these results translate when each of the models' trained parameters are used to predict the labels over the query space, heatmaps of each label were used, to allow easy identification of information such as which areas are particularly domianted by a single habitat, where they are particularly scarce, or areas where multiple habitats co-exist. Full predictions using SVM Regression was ommitted largely due to its poor performance observed from the average error above, in addition to the extensive computational time required for predictions compared to the other three algorithms.
% 
% \begin{figure}[H]
%     \title{\large{\textbf{Linear Regression}}}
%     \centerline{\includegraphics[scale=0.85]{multioutput_distr_colourbar.pdf}}
%     \centerline{\includegraphics[scale=0.85]{multi4_preds_lr.pdf}}
%     \caption{Map of full query Linear Regression 4-label predictions}
%     \label{fig:multi4_lr}
% \end{figure}
% 
% It is immediately apparent that while linear regression has an average error within reason, the maps generated are likely faulty - each label's distribution is almost consistent throughout the $200$km-squared area, a uniformity that is highly improbable if not impossible, given the large area of benthos covered. Moreover, the sum of the normalised predictive distributions at most points sum to almost $2$ - well beyond the expected value, $1$.
% 
% \begin{figure}[H]
%     \title{\large{\textbf{K Nearest Neighbour Regression}}}
%     \centerline{\includegraphics[scale=0.85]{multioutput_distr_colourbar.pdf}}
%     \centerline{\includegraphics[scale=0.75]{multi4_preds_knn.pdf}}
%     \caption{Map of full query K Nearest Neighbour Regression 4-label predictions}
%     \label{fig:multi4_rf}
% \end{figure}
% 
% K-Nearest Neighbour performs slightly better, as there are actually visible differences in the distribution of label 3 throughout the area, whilst showing traces of the other three labels. However, the failure for label distributions per point can be easily observed here, considering that a sizable portion of label 3 areas occur at roughtly a $0.5$ rate, but the combination of other labels in the same space quite clearly don't make up the remaining $0.5$, as they mostly sit near the $0.0$ mark. Moreover, the predictions themselves are very noisy - there are no observable contiguous areas where the occurrence of a habitat occurs at a similar rate, and nor are there visible smooth transitions where a habitat goes from high to low density.
% 
% \begin{figure}[H]
%     \title{\large{\textbf{SVM Regression}}}
%     \centerline{\includegraphics[scale=0.85]{multioutput_distr_colourbar.pdf}}
%     \centerline{\includegraphics[scale=0.85]{multi4_preds_svm.pdf}}
%     \caption{Map of full query SVM Regression 4-label predictions}
%     \label{fig:multi4_svr}
% \end{figure}
% 
% SVM Regression is able to provide more reasonable predictions showing distinct areas of specific habitats that are more `natural' compared to Logistic Regression and K-NN regression thus far. However, considering the rarity of labels $0$ and $1$ in the original data, it is quite unrealistic that they could be almost uniformly present throughout the entire of Scott Reef at such a high rate. Again, visual inspection indicates that a large portion of the map has violated the constraints of the normalised label distributions summing back up to $1$ at each datapoint.
% 
% \begin{figure}[H]
%     \title{\large{\textbf{Random Forest Regression}}}
%     \centerline{\includegraphics[scale=0.85]{multioutput_distr_colourbar.pdf}}
%     \centerline{\includegraphics[scale=0.85]{multi4_preds_rf.pdf}}
%     \caption{Map of full query Random Forest Regression 4-label predictions}
%     \label{fig:multi4_rf}
% \end{figure}
% 
% Random Forest Regression produces a similarly varied map, but very different to that of KNN and SVM regression in some key areas. Labels 1, 2, and 3 have a notably higher predicted occurence rate throughout Scott Reef, with visible swaths that outweigh label $4$ in the region. Without marine biologist expert or similar input, however, it is hard to determine whether the large regions of the less common habitats predicted by the Random Forest Regressor are likely in a location such as Scott Reef.
% 
% As a result of the constraint requiring label distributions per point to sum to $1$, the above use of independent regressors is more of an illustrative exercise than one that can be relied on for real-world use - even if some of the information it provides appears to be of some use. In the extreme cases, some coordinates had no labels at all (the distribtions `summed' to $0$), while others were as high as $3$. These suffer from the same disadvantage as in the previous section on single-output predictions by stating predictions in absolutes rather than providing confidence levels - and again, we apply Gaussian Processes to the problem to attempt to alleviate this, but using regression this time around.
% 
% \subsection{Coercion of Gaussian Process Regression}
% Although a Gaussian Process is only designed to deal with single outputs, each of the label distributions per datapoint are seperate values, meaning it is possible to use multiple Gaussian Processes to allow it to work as a multi-output model. Morever, compared to the previous coercion of deterministic methods, we can use the variance of the Gaussian Process over each label to visualise the uncertainty of predictions. In contrast to previous methods however, due to the established $O(n^3)$ complexity, it was impractical to perform 10-fold cross validation on the full $\geq 5000$ data points in the downsampled training set. Instead, a method similar to \todo{argmax, 230 points per class, 920 total, randomly sample the 920 until predictions on remaining points gives best performance} was used to select the training points.
% 
% % Average error was $0.14409166865082482$ in 10-fold CV \todo{(format)}
% \begin{table}[H]
%     \centering
%     \begin{tabular}{|c|c|}
%         \hline
%         Labels used & Root Mean Squared Error \\\hline
%         4 & 0.14409\\
%         24  & ? \\
%         \hline
%     \end{tabular}
%     \label{table:gpmbasicresults}
%     \caption{Multiple Gaussian Process Regression average error for the two label sets}
% \end{table}
% 
% \Cref{fig:gpm4_simple_preds} below shows the full predictions for the 4-label dataset, with the majority of predictions for labels $0$, $1$, and $2$ lying in the range $[0.1, 0.2]$. This is a likely indicator that the model and its parameters are inappropriate for the problem at hand, despite the variability of label 3 providing a more reasonable prediction of its occurrence. Looking at the actual predicted values, however, reveals how far they have deviated from summing to $1$ per point - \todo{waiting for another prediction to finish to pull values back up}.
% 
% % While the ability of a Gaussian Process allows the data to `speak for itself' aided in bringing the average sum of distributions per point to $1$ compared to the simpler models in \ref{ss:commonMLcoercion}, this property is still not inherently there, not to mention that the full query dataset of just under $3,000,000$ points was too computational intensive to perform full predictions on.
% 
% \begin{figure}[H]
%     \centering
%     \includegraphics[scale=0.85]{gp4_mp_allpreds_colourbar.pdf}
%     \includegraphics[scale=0.85]{gp4_mp_allpreds.pdf}
%     \label{fig:gpm4_simple_preds}
%     \caption{Gaussian Process predictions on full query data}
% \end{figure}
% 
% Since multiple Gaussian Process Classification (Regression) does not discard a majority of the data compared to single-label Gaussian Process Classification, observing the variance is now more meaningful as \textit{each } each of the possible $c$ classes has a probability and corresponding variance of occurrence. However, as we see in \Cref{fig:gpm4_simple_vars}, not much can be gained from the 4-label case, considering that labels 0, 1, 2 have almost no variance, with only label 3 contaiing any visible sections with variance higher than $0.13$, going as high as $0.41$. Given that none of the methods so far are able to simultaneously (at least mostly) adhere to the constraints of the Dirichlet Distribution without explicitly incorporating it or provide realistic predictions given the rich habitat distribution data, we turn to a model that can in theory do both.
% 
% \begin{figure}[H]
%     \centering
%     \includegraphics[scale=0.85]{gp4_mp_vars_all_colourbar.pdf}
%     \includegraphics[scale=0.85]{gp4_mp_vars_all.pdf}
%     \label{fig:gpm4_simple_vars}
%     \caption{Gaussian Process variances on full query data}
% \end{figure}
% 

\subsection{Dirichlet Multinomial Regression}

The last model used was the Dirichlet Multinomial, which incorporates the constraint where predictions over any number of labels had to sum back to $1$, as a result of the Dirichlet distribution component. This means that from a mathematical standpoint, these predictions will be more `correct' for multi-output labels than all the previosly explored models - but we also want to see how they hold up in practice.

To assess initial performance, the weights and hence the $\alpha$ parameter was obtained via the maximum a posteriori estimation.

% \begin{tabular}{|c|c|c|c|c|c|}
%     \hline
%     Algorithm & 10F-CV F1 & 10F-CV Accuracy & Parameters & Data \\\hline
%          DM         &  0.13802716811804644 & 0.37856057852908254 &            &  full labels  \\
%          DM         &   0.287405310254214  &  0.757925654489819  &            & simple labels \\ %     \hline
% \end{tabular}

\begin{table}
    \centering
    \begin{tabular}{|c|c|}
        \hline
        Labels used & Root Mean Squared Error \\\hline
        4 & 0.17664\\
        24  & 0.05916\\
        \hline
    \end{tabular}
    \label{table:dmbasicresults}
    \caption{Dirichlet Multinomial Regression average error for the two label sets}
\end{table}

\begin{figure}
    \begin{minipage}{\linewidth}
        \centerline{\includegraphics[scale=0.85]{dm_standalone_colorbar.pdf}}
        \centerline{\includegraphics[scale=0.85]{dm_simplelabel_heatmaps.pdf}}
        \caption{Distribution heatmaps over each label (in the simple 4-label case) for Dirichlet Multinomial Regressor output on query points}
        \label{fig:dm_4label_heatmap}
    \end{minipage}
    \hfill
\end{figure}

For the simplified labels, we can see some similarities with the models generated using random forests and Gaussian processes (\Cref{fig:det4preds}, \Cref{fig:gpogpe}),  with certain regions matching up to different predictions. For example, all predictions were able to agree on the general dominance of label 3 throughout the reef, but due to the single-output nature of the previous methods, this dominance also meant it was the \textit{only} label to appear in a majority of the predictive maps. Using this model, the actual occupancy rate of label 3 can be determined, instead of only being able to see `all sand'.The ability to quantify the presence of certain habitats also becomes quite trivial, as a single line of Python code can provide information such as label 3 occurring at a rate of more than $0.5$ in $73\%$ of the predicted query space, whereas labels $0$, $1$, and $2$ only occur at a similar rate $4.7\%$, $9.9\%$, and $2.3\%$ of the time.

Using the Dirichlet multinomial that makes full use of the original multi-label data, more detailed observations can also be made regarding the confidence of the multi-output predictions made, by observing the entropy of the Dirichlet distribution using the $\alpha$ parameter \todo{(cite equation)}. Low entropy areas indicate a low variance in the data and hence a confidence in the predictions made in that area, with the reverse being true for high-entropy areas. \Cref{fig:dm4_entropy} shows some key regions (in dark blue) that are very low entropy, with large regions of moderate entropy (cyan) predictions spread out over the query space. Of note is the fact that the noticable areas of co-habitation between labels $1$ and $3$ where they are close to an even $50:50$ split are very low entropy, highlighting the Dirichlet multinomial's ability to detect when a subset of labels within multi-output data correlate with one another with confidence.

\begin{figure}[H]
    \begin{minipage}{\linewidth}
        \centerline{\includegraphics[scale=0.9]{dm4_entropy_map.pdf}}
        \caption{Entropy plot over all query points for the simplified labels}
        \label{fig:dm4_entropy}
    \end{minipage}
    \hfill
\end{figure}

\todo{(does this go here?)}
What becomes apparent is that in the areas where the DM is confident of a mix of certain set of predominant labels, the GP is instead equally uncertain of each of them with a considerably higher variance, which is misleading information when taken at face value. For example, this sort of uncertainty may be taken into consideration purposes, where autonomous vehicles are used to collect data, or in making decisions with regards to conservation efforts. In the first scenario, resources are being wasted on areas where models such as the DM can be confident of a particular distribution of labels, whereas in the second, important conservation actions may be withheld if the \textit{certainty} of information is brought into question. For example, in an area that contains a particular mix of coral and bleached coral, a DM has the potential to make a confident prediction of their coexistence, whereas a GP would make predictions where their respective probabilities in a one-vs-all classifier may be close to their distribution in the area, but have a high noise factor.

% \todo{summarise and plot the variances here}
\subsection{Dirichlet Multinomial Predictive Map Variance}

As the above results from Dirichlet multinomial regressoin above were obtained using the Maximum a Posteriori (MAP) estimate of the parameters underlying the Dirichlet distributions $\alpha$ values, only the single set of optimal parameters were used, with none of those within the rest of the posterior distribution tested. To confirm that the maps generated via optimisation using MAP, Markov Chain Monte Carlo (MCMC) was used to obtain draws of weights from the posterior distribution. The purpose of this was to be able to obtain chains that had reasonably converged, and then observe whether the habitat maps created using all these weights would generally agree on the presence and distribution of habitats. 

To calculate convergence using the Gelman-Rubin r-hat statistic, $3$ MCMC chains were run simultaneously for both the simplified ($4$) labels and the full set of labels ($24$). The former required $4*19=76$ weights (number of labels $\times$ number of features), and the latter $24*19=456$ weights. Due to the large number of dimensions being explored by the MCMC, the $3$ chains in both cases were not able to fully converge to $1.0$. The weights for the simplified labels were close to convergence, with an r-hat score of $1.065$, although $4$ of the $76$ weights were above $1.2$ ($1.45, 3.53, 1.43, 1.21$). The weights for the full $24$ labels fared considerably worse, with an r-hat score of $1.429$, with only $243$ of the $456$ weights being below $1.1$.

% To select an optimal set of parameters for the dirichlet multinomial, Markov Chain Monte Carlo (MCMC) was used to draw samples from the posterior distribution \todo{(refer to equation?)} over $3,000,000$ runs, with the maximum a posteriori estimate used as the starting value for the weights. To select the single best set of weights from the sequence of chains, every single one was evaluated by being used to do Dirichlet Multinomial regression, where the weights that resulted in the lowest predictive variance (average over all variances) was considered to be the best set of parameters. The weights that corresponded to the lowest average variance also corresponded to the lowest average error compared to the original normalised weights. After the $3,000,000$ runs, the MCMC in both cases (the simplified 24 labels, as well as the full set) was considered to have converged, as the Gelman and Rubin ($\hat{r}$) convergence statistics were calculated to be \todo{(? and ?)}, both very close to the ideal value of $1.0$, Furthermore, for the \todo{24-label case}

% MCMC 2 million runs, 24 labels
% 18-smallest variance corresponds to the 1-smallest error - index 1444288 \\
% 19-smallest variance corresponds to the 2-smallest error - index 1444289 \\
% 20-smallest variance corresponds to the 4-smallest error - index 1444292 \\
% 21-smallest variance corresponds to the 0-smallest error - index 1444291 \\
% 22-smallest variance corresponds to the 3-smallest error - index 1444290 \\

% np.save('data/W_2m_1444288', chains[1444288])
% np.save('data/W_2m_1444289', chains[1444289])
% np.save('data/W_2m_1444292', chains[1444292])
% np.save('data/W_2m_1444291', chains[1444291])
% np.save('data/W_2m_1444290', chains[1444290])

\begin{figure}[H]
    \centerline{\includegraphics{dm4_9m_0_mcmc_weight_hist.pdf}}
    \caption{MCMC weights for 4-label, 19-dimension data case \todo{(need to separate this into separate images, possibly remove axis ticks)}}
    \label{fig:4l-mcmc_weights}
\end{figure}

\begin{figure}[H]
    \centerline{\includegraphics{dm24_950k_0_mcmc_weight_hist.pdf}}
    \caption{MCMC weights for 4-label, 19-dimension data case \todo{(need to separate this into separate images, possibly remove axis ticks)}}
    \label{fig:24l-mcmc_weights}
\end{figure}


\pagebreak
\section{Biodiversity}

Another beneficial aspect of Dirichlet Multinomial Regression is that it inherently provides information about the distribution of different habitats in a given region, allowing observations on biodiversity to be made without extra steps such as clustering, which can be prohibitively expensive on datasets with millions of datapoints and tens (or more) of dimensions. Locating the co-existence of certain species would involve searching over the space of predictions for the desired distribution of habitats. 

As the 4-label case already aggregated similar classes from $24$ down to $4$, there was minimial biodiversity to observe over the query space, requiring us to perform predictions over the full $24$ labels to be able to find more abundant occurrences of biodiversity. To give a qualitative visual representation of the distribution of labels over the query space when using all $24$ labels, the predictive heatmaps for each of them is shown in \Cref{fig:dm24_0-7}, \Cref{fig:dm24_8-15}, and \Cref{fig:dm24_16-23}. The numbers in brackets next to each title indicate the simplified label they correspond to.

\begin{figure}
    \includegraphics[clip, width=\columnwidth]{dm24_heatmaps_0-3.pdf}\\
    \includegraphics[clip, width=\columnwidth]{dm24_heatmaps_4-7.pdf}
    \caption{\todo{(need to redo these to show more colour @\_@)}}
    \label{fig:dm24_0-7}
\end{figure}

\begin{figure}
    \includegraphics[clip, width=\columnwidth]{dm24_heatmaps_8-11.pdf}\\
    \includegraphics[clip, width=\columnwidth]{dm24_heatmaps_12-15.pdf}
    \caption{\todo{(need to redo these to show more colour @\_@)}}
    \label{fig:dm24_8-15}
\end{figure}

\begin{figure}
    \includegraphics[clip, width=\columnwidth]{dm24_heatmaps_16-19.pdf}\\
    \includegraphics[clip, width=\columnwidth]{dm24_heatmaps_20-23.pdf}
    \caption{\todo{(need to redo these to show more colour @\_@)}}
    \label{fig:dm24_16-23}
\end{figure}

Given the multi-label distribution predictions, there are any number of ways to either quantatively or qualitatively measure biodiversity. As an instance of the former, strict numerical conditions are used to define biodiversity for the example below. Depending on the aim of a particular environmental study or survey though, the definition of biodiversity is entirely flexible - it may refer to the co-existence of a specific few species or habitats ignoring all others, or it may be general biodiversity that defines cohabitation between any habitats. We take the latter of these formulations to show the power provided by the Dirichlet multinomial. For any given point (and by extension, region, where this pattern occurs frequently enough that the number of points fitting the criteria are dense enough to described as a cluster), $n$ number of labels are said to co-occur if for a given data point, the relevant area of benthos corresponding to a coordinate contains more than $n$ labels (number of `co-existing' habitats) that occur at a rate of at least $e$, where $0 < e \leq 1$. Given this specific `version' of biodiversity, $e$ would naturally need to decrease as $n$ increases - while it is possible to have large areas where at least two labels occur at a higher rate than $0.15$, this is of course not possible for seven points, for example ($7\times0.15=1.05$).

% \todo{[PLACEHOLDER]}
% We can visually observe that labels $7, 8, 10$ (variants of sand) occur in the same regions, with the following occupancy rates, beyond areas containing only negliglble traces ($\equiv0.00-0.02$) $7$ at a $\approx0.12$ occpuance rate, $8$ at $\approx0.18-0.20$, and $10$ at $\approx.17$.

\begin{figure}
    \makebox[\textwidth]{
        \includegraphics[scale=0.5]{dm24_cohab_map_2habs_143902points.pdf}
        \includegraphics[scale=0.5]{dm24_cohab_map_3habs_151814points.pdf}
    }
    \makebox[\textwidth]{
        \includegraphics[scale=0.5]{dm24_cohab_map_4habs_144647points.pdf}
        \includegraphics[scale=0.5]{dm24_cohab_map_5habs_134044points.pdf}
    }
    \makebox[\textwidth]{
        \includegraphics[scale=0.5]{dm24_cohab_map_6habs_123520points.pdf}
        \includegraphics[scale=0.5]{dm24_cohab_map_7habs_110622points.pdf}
    }
    \caption{Cohabitation for the stated occupancy threshold for the label of each plot, from $2-7$.}
\end{figure}

\begin{figure}
    \makebox[\textwidth]{
        \includegraphics[scale=0.5]{dm24_cohab_map_8habs_78805points.pdf}
        \includegraphics[scale=0.5]{dm24_cohab_map_9habs_57530points.pdf}
    }
    \makebox[\textwidth]{
        \includegraphics[scale=0.5]{dm24_cohab_map_10habs_42303points.pdf}
        \includegraphics[scale=0.5]{dm24_cohab_map_11habs_27102points.pdf}
    }
    \makebox[\textwidth]{
        \includegraphics[scale=0.5]{dm24_cohab_map_12habs_17774points.pdf}
        \includegraphics[scale=0.5]{dm24_cohab_map_13habs_9711points.pdf}
    }
    \caption{Cohabitation for the stated occupancy threshold for the label of each plot, from $8-13$.}
\end{figure}

For the definition of biodiversity taken in the above plots, it is easy to observe the biodiversity over the 24 labels in general, with a general trend that biodiveristy between an increasing number of habitats, even with the decreasing threshold, results in a lower density of biodiverse areas. The initial threshold when considering $2$ labels was $0.15$, dropping by $10\%$ (or $0.9$ of the previous value) for each additional label considered. For these particular parameters, any signs of visibly continuous areas of biodiversity disappear when considering more than $13$ labels.

This approach can be easily modified to target the tracking of specific habitats - for example, if data is collected periodically in any area where coral bleaching is suspected to occur, predictive maps can be generated at the same intervals, allowing the \textit{changes} in biodiversity to be observed. Although such a task can be more generally achieved even using deterministic methods by simply taking a count of labels over the query space to see positive/negative trends, this is quite coarse in comparison, and specifying with accuracy the regions where change occurred involves extra work - for example, to assess if an area that was previously unbleached coral had become bleached through automated means, an individual working with the data may have to define an algorithm that checks the area around points up to a certain distance, then observe any potential changes in predictions generated from newly collected data, requiring considerably more effort in designing a method that allows analysis to be done efficiently. In contrast - performing the checks for general biodiversity using the above biodiversity across all 24 labels only took $11$ seconds, with checks for individual labels take less than $1$ second.

% \includegraphics{dm24_cohab_map_14habs_6928points.pdf}
% \includegraphics{dm24_cohab_map_15habs_3452points.pdf}
% \includegraphics{dm24_cohab_map_16habs_1646points.pdf}
% \includegraphics{dm24_cohab_map_17habs_1561points.pdf}
% \includegraphics{dm24_cohab_map_18habs_1893points.pdf}
% \includegraphics{dm24_cohab_map_19habs_2208points.pdf}
% \includegraphics{dm24_cohab_map_20habs_1869points.pdf}
% \includegraphics{dm24_cohab_map_21habs_703points.pdf}
% \includegraphics{dm24_cohab_map_22habs_569points.pdf}
% \includegraphics{dm24_cohab_map_23habs_156points.pdf}

\input{experiments/limitations.tex}

