\chapter{Mathematical Background} \label{chap:mathbg}

\section {Dirichlet Multinomial Regression}

Dirichlet multinomial regression, as the name suggestions, combines dirichlet and multinomial distributions to achieve the combined model. In particular, we are interested in modeling a distribution over category counts, as there exists relationship in our data such that every bathmetry point corresponds to a certain count of each possible label in the relevant area of benthos. \todo{explain why we should first revisit dirichlet, multinomial distributions separately before looking at dirichlet multinomial regression}

\subsection{Multinomial Distribution}
\todo {equations, description}

\subsection{Dirichlet Distribution}
\todo{descriptions}

$$\theta \sim Dir(\alpha) \text{ , dirichlet distributed random variable}$$ 
$$p(\theta)= \frac{1}{\beta(\alpha)} \Pi_{i=1}^n \theta_i^{\alpha_i - 1} I(\theta \in S) \text{ density function, I is indicator function}$$ 
$$ \theta = (\theta_1, ..., \theta_n), \alpha = (\alpha_1,...,\alpha_n), \alpha_i > 0 \text{ theta - n-dimensional vectors, alpha - parameters for distribution}$$
$$ S = \{x \in R^n : x_i \geq 0, \sum x_i = 1\} \text{ S is probability simplex, the set of pmfs on numbers 1 through n}$$ 
$$\frac{1}{\beta(\alpha)} = \frac{\Gamma(\alpha_0)}{\Gamma(\alpha_i) ... \Gamma(\alpha_n)}, \alpha_0 = \sum_{i=1}^n \alpha_i \text{ generalised beta function}$$

\subsubsection{Dirichlet Multinomial Regression}

\todo{descriptions}

$$DM(C|\alpha) = \frac{M!}{\Pi_k C_k!} \frac{\Gamma(\sum_k \alpha_k)}{\Gamma(\sum_k c_k + \alpha_k)} \Pi_{k=1}^k \frac{\Gamma(C_k + \alpha_k)}{\Gamma(\alpha_k)}$$
$$ M = \sum_k c_k $$

For the regressor, the two activation functions that were considered were exponential and softmax, where the former often provided better mapping predictions, but the latter is preferable in the general case due to its better numerical stability \todo {include graphs of exponential and softmax here}.
$$\alpha_k = \exp\{x^T w_k\}$$
$$\alpha_k = \text{softmax}\{x^T w_k\}$$

The weights $w$ here are in fact a matrix of weights with dimensions $(K \times D)$, where $K$ is the number of possible labels across the dataset, and $D$ is the dimensionality of the dataset. Muliplying the dirichlet multinomial prior by the likelihood then then gives the equation over which to optimise to predict the normalised label counts at any given point.

This gives the joint-log-likelihood over both the dirichlet and multinomial distributions:
\begin{multline}
    \sum^N_{n=1} [\log(M_k) - \sum_k \log(c_k!) + \log \Gamma(\sum_k \alpha_k(x_n)) - \log \Gamma(\sum_k c_{nk} + \alpha_k(x_n))] \\
    + \sum^N_{n=1} \sum^K_{k=1} [\log \Gamma(c_k + \alpha(x)) - \log \Gamma(\alpha_k(x_n))] \\
    + \sum^K_{k=1} [-\frac{\phi}{2} \log(2\pi \phi) - \frac{1}{2}w_k^T \phi \mathbb{I} w_k]
\end{multline}

To optimise this equation, the partial derivative of the above over the weights $w$ are considered:
\begin{multline}
    \partial \frac{\log p(c, x)}{\partial w_k} = \sum_{n=1}^N x_n \alpha_k (x_n) [\psi(\sum_l \alpha_l(x_n)) - \psi(\sum_k c_{nk} + \alpha_k(x_n))] \\
    + \sum^N_{n=1} x_n \alpha_k (x_n) [\psi (c_{nk} + \alpha_k(x_n)) - \psi(\alpha_k(x_n))] - \frac{1}{\phi} w_k
\end{multline}

\todo {explain all the symbols here}

\subsection{Using MCMC instead of MAP}
\section{Illustrative Example}

The differences between a Gaussian Process that provides the probability distribution of possible labels compared to the Dirichlet Multinomial Regressor that provides the distribution of actual labels at a point, are highlighted in the illustrative example below. Note that three clusters were synthesised, with clusters A, B containing $0.7:0.3$ and $0.3:0.7$ average ratios in label mix per point respectively, while cluster C contained an even $0.5:0.5$ average split, where cluster had $100$ points. The colours on the overall plot are only representative of the \textbf{most} common label at each point - the actual distributions at each point are shown in the graphs following it.

% \todo{generate toy example with a mixed label A,B region and separate regions of mostly A, mostly B respectively}

\begin{figure}[H]
    \includegraphics[scale=0.7]{toydataplot.pdf}
    \caption{Plots of the three clusters, with labels taking on the argmax of each point}
    \label{fig:toyplot}
\end{figure}

\begin{figure}[H]
    \includegraphics[trim={0 2cm 0 2cm}]{toyplot_hist_distr_legend.pdf}
    \caption{Legend/axes for the following histogram plots showing distribution of labels at each point}
    \label{fig:toyplothist_legend}
\end{figure}

\begin{figure}[H]
    \begin{minipage}{.49\linewidth}
        \includegraphics[width=\linewidth]{toy_clusterA_distr.pdf}
        \caption{Label distribution of cluster A}
        \label{fig:toyclusterA}
    \end{minipage}
    \hfill
    \begin{minipage}{.49\linewidth}
        \includegraphics[width=\linewidth]{toy_clusterB_distr.pdf}
        \caption{Label distribution of cluster B}
        \label{fig:toyclusterB}
    \end{minipage}
    \begin{minipage}{.49\linewidth}
        \includegraphics[width=\linewidth]{toy_clusterC_distr.pdf}
        \caption{Label distribution of cluster C}
        \label{fig:toyclusterC}
    \end{minipage}
\end{figure}

In this example, the GP and DM models were each trained on half of each cluster, and made to predict the other half. However, as a standard GPC can only have single label inputs and outputs, a approximation/simplification was made for the purpose of calculating average error, whereby the label was simply taken to be the most frequently occurring label at any given point. While this is a reasonable simplification for clusters A, B as the dominant label has majority share, this is not the case for C, as the split between the two labels per point in the cluster is exactly even. In an initial attempt to counter this, multi-task GPs were considered as a means of making a \textit{fairer} comparison between a GP and DM, but the idea was ultimately discarded as it was not fit for purpose, one of the primary issues being that the model does not inherently restrict the outputs of a given datapoint to sum to $1$, instead being at the mercy of the parameters of the GP. 

\subsection{Results}

The results and plots for this example are below, and figures displayed were taken from an average of $20$ runs.

\begin{tabular}{|C{4cm}|C{4cm}|c|}
    \hline
    & Dirichlet Multinomial Regression RMSE* & Gaussian Process Classifier (argmax) RMSE \\\hline
    Original data & 0.070179271314358999 & 0.26833333333333337 \\\hline
    Quadratic-space projection & 0.065630111843395234 & 0.43433333333333335 \\\hline
    Cubic-space projection & 0.29019235800882354 & 0.43725490196076466 \\\hline
\end{tabular}
\begingroup
\tiny{RMSE - root mean squared error}
\endgroup

As can be seen from the above overvise, the DM performed best when projecting the data to quadratic space, while the GPC didbest on the original data as-is. This was taken into account for the plots below for the DM and GP respectively, which used an instance of the more favourably performing processed data. Note that the exact probabilities provided by the GP are hown in the following plots, in contrast to the argmax taken for error-calculation purposes.

\begin{figure}[H]
    \includegraphics[scale=0.6]{toy_dm_pred_plot0.pdf}
    \caption{DM Label distribution of label 0}
    \label{fig:toylabel0}
\end{figure}
\begin{figure}[H]
    \includegraphics[scale=0.6]{toy_dm_pred_plot1.pdf}
    \caption{DM Label distribution of label 1}
    \label{fig:toylabel1}
\end{figure} 
%\todo{dm - sitting on averages in each cluster. cause for concern?}

\begin{figure}[H]
    \includegraphics[scale=0.6]{gp_with_vars0.pdf}
    \caption{OvR GP performance with variance for label 0}
    \label{fig:toylabel0}
\end{figure}
\begin{figure}[H]
    \includegraphics[scale=0.6]{gp_with_vars1.pdf}
    \caption{OvR GP performance with variance for label 1}
    \label{fig:toylabel1}
\end{figure} 

As we can see, the DM performed notably better than the GP, with the predictions in each cluster staying true to the mean values. The GP, despite the $\sim 0.26$ error rate, can be seen to follow a consistent trend in the test data instead of adjusting to the two different classes. Most importantly, for both label 0, 1, the DM identifies that the portion of either label in cluster C is 0.5, due to its ability to learn and predict distributions at a point. As GPs are not designed to and cannot model this, it instead finds that neither label has a high probability, whilst also having a very high variance (beyond the $[0, 1]$ bound which again, the DM enforces but is not a property of a GP).

However, this is admittedly admittedly a rather simple example that assumes we have a sufficient amount of data from the \textit{three} possible habitat clusters - A by itself, B by itself, and a homogeneous mix of A and B, and as such, more detailed comparisons will be made using the full training dataset.

From this basic example, it is apparent that in the area where there is an even mix of labels A, B, the Gaussian Process' predictions are both noisy and very uncertain about their predictions, where human intervention would be required to observe the fact that it is in fact a consistent mix of both. In contrast, the dirichlet multinomial regressor is more confident in the fact that that area does in fact have a mix of labels.

