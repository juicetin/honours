\chapter{Probabilistic Habitat Mapping} \label{chap:gps}

The methods of habitat mapping explored until now were mostly deterministic ones, where predictions were absolute, and as such did not provide a \textit{level of confidence} in the predictions made, or in other words, probabilistic output. The exception to this was logistic regression, but even then, as a parametric method, the complexity of the model must be defined beforehand, whereas a Gaussian Process in simple terms allows the data to 'speak for itself`. More formally, this refers to a Gaussian Process' non-parametric nature, meaning the data is incorporated directly into the model where new data can increase the confidence of the model.

In this chapter, we will look at Gaussian Processes as technique to generate predictive habitat maps. We begin by visting Gaussian Process Regression, and how a small extension/post-processing step extends it to allow Gaussian Process Classification. Of the different ways to train the hyperparameters of Gaussian Processes, the one selected was Leave-One-Out Cross Validation (LOO-CV). They also need to use a \textit{kernel} that defines the relationship between any two points, forming the full covariance matrix - the one chosen was the squared exponential kernel, explained in detail in the following sections. Note that detailed proofs and derivations are not covered here, and interested readers should consult Rasmussen and William's Gaussian Processes for machine Learning \cite{rasmussen06} for a definitive guide to all things Gaussian Process related. In particular, Chapters 2 and 5 are of the most relevance, as they detail Gaussian Process Regression, and Model Selection and Adaptation of Hyperparameters respectively.

% $\mathbf{f_*}$ - predictive function outputs corresponding to input data $X_*$

\section{Gaussian Process Regression}\label{chapsec:gpr}

Compared to standard linear regression that explains data by optimising $\mathbf{y=X\beta} + \epsilon$, where $\mathbf{y}$ are the response variables, $\mathbf{X}$ are the input variables, and $\mathbf{\beta}$ are the regression coefficients, Gaussian process regression takes a Bayesian approach by adjusting probabilities when given more information (input data), and performs inference over functions.

We define a Gaussian Process on input $\mathbf{x}$ to have mean ($m$) and covariance ($k$), where $\mathbf{x}$ and $\mathbf{x'}$ are the training and test inputs respectively:
\begin{equation}
f(x) \sim GP(m(\mathbf{x}), k(\mathbf{x}, \mathbf{x'}))
\end{equation}

The chosen kernel is the squared exponential (on the right hand side). The base covariance function between points $p, q$, where $\mathbf{x_p, x_q}$ are the vector of $n$ features at each point is thus given by:
\begin{equation}\label{eq:simplegpcov}
    cov(f(\mathbf{x_p}), f(\mathbf{x_q)}) = k(\mathbf{x_p, x_q}) = \exp(-\frac{1}{2}|\mathbf{x_p}-\mathbf{x_q}|^2)
\end{equation}

From the above equation, it is evident that points that are very close together in the $n$-dimensional input space would have a covariance of $1$ (as $\lim_{\mathbf{x_p=x_q}} exp(-\frac{1}{2}|\mathbf{x_p-x_q}^2| = 1$) - when assuming all features are equally important and correlated when assessing their distance. Because logic would define that such an assumption is unlikely to hold with real world data, a \textit{lengthscale} needs to be applied to each dimension to give important features more weight, and reduce the impact of less significant features on the covariance between two points. The vector of lengthscales would then be optimised along with the other parameters when training the Gaussian Process model. Should \ref{eq:simplegpcov} prove to be the optimal setup, the length scale vector would then simply comprimse of $1$s after training. The updated covariance function $k$ would then be:
\begin{equation}\label{eq:fullgpcov}
    k\mathbf{(x_p, x_q)} = \sigma^2_f \exp(-\frac{1}{2l^2} (\mathbf{x_p-x_q})^2) + \sigma^2_nI
\end{equation}
where $\sigma_f$ is the variance in the training data, and $\sigma_n$ is the variance of the Gaussian noise. \todo{clarify what's happening here}

To allow simplifications of notation in the following equations, we define some abbreviations related to \autoref{eq:fullgpcov} depending on what data is involved in the covariance matrix. 

\todo{(lay out these abbreviations nicely)}
To indicate the full covariance matrix over training points: $K = K(X, X)$ \\
To indicate the full covariance between training points and test points: $K_* = K(X, X_*)$
To indicate the covariance between a single test point with all training points: $\mathbf{k_*} = \mathbf{k(x_*}, X)$

By conditioning the joint Gaussian prior distribution on the observed data, we obtain our predictions at test points \todo{(EXPAND. Possibly include non-abberviated version first for clarity)}:
\begin{equation}
    \mathbf{f_*} | \mathbf{x_*}, X, \mathbf{f} \sim \mathcal{N}( \mathbf{k_*} K^{-1} \mathbf{f},  
    k(\mathbf{x_*, x_*}) - \mathbf{k_*} K^{-1} K_*)
\end{equation}

Taking the first part of the Gaussian Distribution, the mean, and the second, the variance, we obtain our predictions and variance on a single test point:
\begin{equation}
    \bar{f_*} = \mathbf{k_*}^T(K+\sigma_n^2)^{-1} \mathbf{y}
\end{equation}
\begin{equation}
    \mathbb{V}[f_*] = K(\mathbf{x_*},\mathbf{x_*}) - \mathbf{k_*}^T (K+\sigma^2_n)^{-1}\mathbf{k_*}
\end{equation}

In practice, a test dataset would not be calculated one test point at a time as the above equation suggests, but all at once - to do so simply requires taking the covariance between all test points and training points whenever covariance of only a single test point with all training points is involved. These equations are used on the basis that all their parameters have already been determined - the most common process for doing this and the one used in this study is to maximise the log marginal likelihood. Although optimising the log marginal likelihood is required to use the above equations, its representation requires notation defined above, and as such formalising this aspect has been withheld until this point. However, because a slightly customised log likelihood function is used in Leave-ont-out Cross Validation (LOO-CV) below, we will only cover the standard marginal likelihood used by Gaussian Processes briefly here.

The marginal likelihood for this model can be obtained by integrating the likelihood times the prior with respect to function values $\mathbf{f}$:
\begin{equation}
    p(\mathbf{y}|X) = \int p(\mathbf{y|f}, X)p(\mathbf{y}|X)d\mathbf{f}
\end{equation}

Taking the log of this then gives:
\begin{equation}
    \log p(\mathbf{f}|X) = -\frac{1}{2} \mathbf{f}^T K^{-1} \mathbf{f} - \frac{1}{2} \log|K| - \frac{n}{2} \log 2\pi
\end{equation}

As this equation has an analytical first derivative which can be used to speed up optimisation so long as a software package is used where the optimisation algorithms allows a derivative (also known as a Jacobian), we wish to also know its derivative:
\begin{equation}
    \frac{\partial}{\partial\theta_j} \log p(\mathbf{y}|X,\mathbf{\theta}) = \frac{1}{2} \text{tr}((\mathbf{\alpha\alpha}^T - K^{-1}) \frac{\partial K}{\partial \theta_j}) \text{ where } \mathbf{\alpha} = K^{-1}\mathbf{y}
\end{equation}

Keeping in mind that $K$ initially contains a number of initially unknown parameters ($\sigma_n, \sigma_f, l$), we can then optimise over this log marginal likelihood function as a whole using its derivative to search the multi-dimensional space. However, a variant of this will be used, as explained in the following section.

% * distribution over functions
% \begin{equation}
% \mathbf{f_*} | X, \mathbf{y}, X* \sim \mathcal{N}(\mathbf{\bar{f_*}}, \text{cov}(\mathbf{f_*})) \text{, where}
% \end{equation}
% \begin{equation}
% \mathbf{\bar{f_*}} \triangleq E[f_*|X,\mathbf{y},X_*] = K(X_*, X)[K(X,X) + \sigma^2_n]^{-1}\mathbf{y}
% \end{equation}
% \begin{equation}
%     cov(\mathbf{f_*}) = K(X_*,X_*) - K(X_*, X)[K(X,X) + \sigma_n^2]^{-1}K(X,X_*)
% \end{equation}
% 
% To simplify following notation, we make the following abbreviations:
% $K = K(X, X)$, $K_* = K(X, X_*)$ for full covariances over training and test points, and $k(X, x_*) = k_*$ for the covariance between a single test point and all training points.

\section{Leave-One-Out Cross Validation}

\todo{(expand section)}

To train our data, we chose the extreme case of of cross-validation for model training, where the number of folds used, $k$, is equal to the number of datapoints. By optimising over the sum of cross-valiated log likelihoods, it is no longer strictly only assessing the log marginal likelihood, instead acting as more of a pseudo-likelihood. Directly optimising over the marginal likelihood provides the probability of observed data \textit{given model assumptions}, whereas the cross-validation approach provides the log predictive probability estimates independent of the fufilment of said model assumptions. The latter case is preferrable here as biological experts were not consulted for the duration of the study, meaning some assumptions could have been tuned more accurately if external help was available.

The log probability ommitting training case $i$

$$\log p(y_i|X, \mathbf{y_i}, \theta) = -\frac{1}{2}\log\sigma^2_i - \frac{(y_i - \mu_i)^2}{2 \sigma^2_i} - \frac{1}{2}\log2\pi$$

$$ L_{LOO}(X, y, \theta) = \sigma^n_{i=1} \log p(y_i, X, \mathbf{y_i}, \mathbf{\theta})$$

LOO-CV predictive mean and variance
$$\mu_i= y_i - [K^{-1}\mathbf{y}]_i / [K^{-1}]_{ii} \text{ and } \sigma_i^2 = 1/[K^{-1}]_{ii}$$

partial derivatives with respect to the hyperparameters
$$\frac{\partial{u_i}}{\partial{\theta_j}} = \frac{[Z_j \alpha]}{[K^{-1}]_{ii}} - \frac{\alpha_i[Z_j K^{-1}_{ii}]_{ii}}{[K^{-1}]^2_{ii}}$$
$$\frac{\partial{\sigma_i^2}}{\partial{\theta_j}} = \frac{[Z_jK^{-1}]_{ii}}{[K^{-1}]^2_{ii}}$$

where $\alpha = K^{-1}\mathbf{y}$ and $Z_j = K^{-1} \frac{\partial{K}}{\partial{\theta_j}}$


\section{Gaussian Process Classification} \label{chapsec:gpc}

To perform Gaussian Process Classification, multiple Gaussian Process Regressors are used for each possible label using a one-vs-all approach as in Example \autoref{ex:one-vs-all}, and every label at each data point is assigned both a probability and variance. To ensure the predictions at each point for each label is propertly constrained to be in the range $[0,1]$, we again pass predictions through the logistic sigmoid function (\autoref{eq:logistic}).

\todo{(expand)}

% \section{Subsampling for Gaussian Process experiments}
% \todo{(may need to take this section out - GPy wrapper deals with thousands of points no problem, no longer need to subsample with only 4000 after downsampling the data)}
% 
% Due to the $O(n^3)$ complexity of training a Gaussian Process Classifier, using all $16502$ points was infeasible, so it was necessary to use only a subsample of the training data. As can be seen in the above histograms \todo{(reference the figure instead. may need to combine them into one)}, the distribution of classes in both the simplified and non-simplified versions was very uneven. As a result of this skew, randomly sampling the the training data to fit our GP classifier against resulted in worse results than samplying an equal \textit{number} of points for each class. To obtain a reasonably well-performing set of 1000 points (the number chosen to obtain a balance between performance and time required), 10-fold cross validation was performed on random subsets of this size. To obtain the 1000 datapoints, both stratified sampling, as well as obtaining ratios of labels matching those in the training set were used. After 200 runs, the set with the best performance was used for the remaining GP experiments.

\section{Gaussian Process Approximation} \label{chapsec:gp-approx}

However, significant limitations exist in terms of the number of training datapoints used when using a single Gaussian Process Classifier. To counter this, there are approximation methods which allow more points to be used whilst not expending more time. There are two common ways to do this, the first being using inducing inputs, where approximations are made such that neither extra time or computational power is needed to encapsulate the information of more points. The second are \textit{ensemble} methods - combining the results of several independent Gaussian Processes trained in parallel (at the expense of more computational power), and was the one tested as a part of this study.

\subsection{Product of Experts and Variations}
\todo{(EXPAND explanations in this section)}

Product of GP Experts
\begin{equation}
    \mu_*^{poe} = (\sigma_*^{poe})^2 \sum_k \sigma_k^{-2} (\mathbf{x_*}) \mu_k (\mathbf{x_*})
\end{equation}
\begin{equation}
    (\sigma_*^{poe})^{-2} = \sum_k \sigma_k^{-2} (\mathbf{x_*})
\end{equation}

Generalised Product of GP Experts
\begin{equation}
    \mu_*^{gpoe} = (\sigma_*^{gpoe})^2 \sum_k \beta_k \sigma_k^{-2} (\mathbf{x_*}) \mu_k (\mathbf{x_*})
\end{equation}
\begin{equation}
    (\sigma_*^{gpoe})^{-2} = \sum_k \beta_k \sigma_k^{-2} (\mathbf{x_*})
\end{equation}
The value of each $\beta_k$ is flexible, but as scaling Gaussian Processes to large datasets isn't the primary focus of this study, we simply set each $\beta_k$ to $\frac{1}{M}$, where $M$ is the number of experts, as suggested in \citep{deisenroth15} to maintain reasonable margins of error.

\subsection{Bayesian Committee Machines and Variations}

Bayesian Committee Machine
\begin{equation}
    \mu_*^{bcm} = (\sigma_*^{bcm})^2 \sum_{k=1}^M \sigma_k^{-2} (\mathbf{x_*}) \mu_k (\mathbf{x_*})
\end{equation}
\begin{equation}
    (\sigma_*^{bcm})^{-2} = \sum_{k=1}^M \sigma_k^{-2} (\mathbf{x_*}) + (1-M)\sigma_{**}^{-2}
\end{equation}
where $\sigma_{**}^{-2}$ is the prior precision of $p(f_*)$, which itself is the inverse of the prior variances.

Robust Bayesian Committee Machine
\begin{equation}
    \mu_*^{rbcm} = (\sigma_*^{rbcm})^2 \sum_k \beta_k \sigma_k^{-2} (\mathbf{x_*}) \mu_k (\mathbf{x_*})
\end{equation}
\begin{equation}
    (\sigma_*^{rbcm})^{-2} = \sum_k \beta_k \sigma_k^{-2} (\mathbf{x_*}) + (1-\sum_{k=1}^M \beta_k) \sigma_{**}^{-2}
\end{equation}
where each $\beta_k$ follows the same rules as for the Product of Experts and its variations.

\citep{deisenroth15}

