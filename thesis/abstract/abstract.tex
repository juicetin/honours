\chapter*{Abstract}

Being able to predict the state of benthic habitats based on limited information is crucial for environmental conservation, particularly as the impact of human activity on our oceans is greater than ever before. A fairly large portion of work done in the area uses deterministic methods that strictly assign only one label to a given bathymetry data point, while more advanced models provide probabilistic results over all possible labels at any one point, but similarly only representing a single output. However, like the majority of real life classification problems, habitat mapping is intrinsically a multi-label problem for any data collected at a resolution low enough to be economically feasible to be performed at a large scale. With the motivation of working with (relatively) low resolution bathymetry data, we explore the advantages of treating benthic habitat mapping as a multi-label problem compared to combinations of deterministic, probabilistic, and single-label methods used in previous works.
