\chapter{Background} \label{chap:mathbg}

This chapter will go through in detail the two key methods applied in this thesis - Gaussian process classification, 
and Dirichlet multinomial regression. Gaussian process classification is able to provide probabilistic outputs over the space over possible labels, with variances on each prediction reprsenting the spread of possible predictive values. Dirichlet multinomial regression, on the other hand, allows regression to be performed over multiple outputs that represent a distribution of possible labels, where any given output vector adds up to $1$.

In each of these methods, the motivation for applying it to benthic habitat mapping is briefly explained, followed by explanations the mathematical theory involved, as well as examples to illustrate their behaviour on simple $2$D datasets before applying them to the Scott Reef data in \Cref{chap:experiments}, where the increased number of dimensions prevents makes it harder to visualise any phenomena that occur.

%%%%% GP
\chapter{Probabilistic Classification for Large Datasets}\label{chap:gps}

The methods of habitat mapping explored until now were mostly deterministic ones, where predictions were absolute, and as such did not provide a \textit{level of confidence} in the predictions made, or in other words, probabilistic output. When using information provided by predictive benthic maps, decisions being made will often have far-reaching impacts, and as such all parties involved need to be as well informed as possible. Machine learning techniques such as SVMs cannot provide probabilistic outputs, so despite being popular in machine learning literature, is not the best fit for modeling ocean habitats if intended for real world use. Gaussian processes are state of the art in terms of probabilistic modeling, and a method that has seen some attention in the area of benthic habitat mapping, though it is still considerably less common than more traditional methods like SVMs and random forests.

In this section, we will look at Gaussian processes as a technique to generate predictive habitat maps. We begin by going over the basics of Gaussian process regression, and how a small extension/post-processing step extends it to allow Gaussian process classification. Due to the covariance matrix inversions needed during the training stage, Gaussian processes have an $O(n^3)$ complexity, severely restricting the number of points we can work with. Approximation methods are explored as a way to overcome this constraint, and their performance as well as runtime required to fit models and predict data is compared to a normal Gaussian process. Detailed proofs and derivations are not covered in this chapter, and interested readers should consult Rasmussen and William's Gaussian Processes for Machine Learning book ~\citep{rasmussen06} for a definitive guide to all things Gaussian process related. In particular, Chapters 2 and 5 are of the most relevance, as they detail Gaussian process regression, and model selection and adaptation of hyperparameters respectively.

% $\mathbf{f_*}$ - predictive function outputs corresponding to input data $X_*$
 \section{Gaussian Process Regression}\label{chapsec:gpr}

Compared to deterministic methods like linear regression\footnote{Strictly speaking, linear regression is a Gaussian process with a linear kernel - but due to its simplicity, and that Gaussian processes are not usually explained by weight coefficients but by their hyperparmeters, we are juxtaposing linear regression to Gaussian processes here as a simpler approach.} that explains data by optimising $\mathbf{y=X\beta} + \epsilon$, where $\mathbf{y}$ are the response variables, $\mathbf{X}$ are the input variables, and $\mathbf{\beta}$ are the regression coefficients, Gaussian process regression takes a Bayesian approach by adjusting probabilities when given more information (input data), and performs inference over functions.

We define a Gaussian process on input $\mathbf{x}$ to have a mean function $m$ and covariance function $k$ (or in other words, the kernel), where $\mathbf{x}$ and $\mathbf{x'}$ are the training and test inputs respectively:
\begin{equation}
f(x) \sim GP(m(\mathbf{x}), k(\mathbf{x}, \mathbf{x'}))
\end{equation}

The chosen kernel is the squared exponential, represented by the covariance function between points $p, q$, where $\mathbf{x_p, x_q}$ are the vector of features at each point is thus given by:
\begin{equation}\label{eq:simplegpcov}
    cov(f(\mathbf{x_p}), f(\mathbf{x_q)}) = k(\mathbf{x_p, x_q}) = \exp(-\frac{1}{2}|\mathbf{x_p}-\mathbf{x_q}|^2)
\end{equation}

The free parameters involved in the squared exponential kernel are the length scales $l$, signal variance $\sigma_f$ and noise variance $\sigma_n$, and need to be optimised to perform predictions on unseen data.
\begin{equation}\label{eq:fullgpcov}
    k\mathbf{(x_p, x_q)} = \sigma^2_f \exp(-\frac{1}{2l^2} (\mathbf{x_p-x_q})^2) + \sigma^2_nI
\end{equation}
where $\sigma_f$ is the variance in the training data, and $\sigma_n$ is the variance of the Gaussian noise. The length scales $l$ are not a single variable as the equation may suggest, but a vector of length scales equal in length to the number of dimensions in the inputs $\mathbf{x}$. If $l$ was simply a vector of 1s, which would give every feature in the input space equal weighting, but it is not likely in a real world dataset for every feature to have equal importance. This is what the length scales account for - by tuning $l_i$ for each feature $i$, the model can learn during the fitting process which features are important, and which ones are not. 

 The equations for the predictive means and variances then incorporate the covariance function, and hence the hyperparameters, as follows:
\begin{equation}
    \bar{f_*} = \mathbf{k_*}^T(K+\sigma_n^2)^{-1} \mathbf{y}
\end{equation}
\begin{equation}
    \mathbb{V}[f_*] = k(\mathbf{x_*},\mathbf{x_*}) - \mathbf{k_*}^T (K+\sigma^2_n)^{-1}\mathbf{k_*}
\end{equation}
where $K = K(X, X)$ is the covariance matrix over all training points, and $\mathbf{k_*} = \mathbf{k(x_*}, X)$ is the covariance between a single test point with all training points.

% To allow simplifications of notation in the following equations, we define some abbreviations related to \autoref{eq:fullgpcov} depending on what data is involved in the covariance matrix. 
%  
%  \todo{(lay out these abbreviations nicely)}
%  To indicate the full covariance matrix over training points: $$K = K(X, X)$$
%  To indicate the full covariance between training points and test points: $$K_* = K(X, X_*)$$
%  To indicate the covariance between a single test point with all training points: $$\mathbf{k_*} = \mathbf{k(x_*}, X)$$

Data was generated in \Cref{fig:gphp_ex} below to illustrate a Gaussian process's behaviour when different hyperparameters (as shown below each plot) are used to perform regression. The key things to observe in these 2D examples are that the noise and length scale govern the complexity of the model - both a low length length scale and error are required to create a Gaussian process that attempts to adhere to points as closely as possible, though this can result in `overfitting' of the data, seems to be the case in plot 3.

\begin{figure}
    \centering
    \begin{minipage}{0.53\linewidth}
        \includegraphics[width=0.9\linewidth]{gp_with_variance_plot3.pdf}
        \caption*{(1) $\sigma_f = 2.13$, $l = 1.27$, $\sigma_n = 0.17$}
    \end{minipage}
    \hfill
    \begin{minipage}{0.47\linewidth}
        \includegraphics[width=0.9\linewidth]{gp_with_variance_plot1.pdf}
        \caption*{(2) $\sigma_f = 1$, $l = 1.7$, $\sigma_n = 0.1$}
    \end{minipage}
    \begin{minipage}{0.47\linewidth}
        \includegraphics[width=0.9\linewidth]{gp_with_variance_plot2.pdf}
        \caption*{(3) $\sigma_f = 1$, $l = 0.3$, $\sigma_n = 0.001$}
    \end{minipage}
    \label{fig:gptoyplots}
    \caption{Examples of a Gaussian process' behaviour with different hyperparameters.}
    \label{fig:gphp_ex}
\end{figure}


\section{Leave-One-Out Cross Validation}

To determine the hyperparameters of the training data, cross-validation for model training is used, with the number of folds used, $k$, equal to the number of datapoints, only excluding one data point per round - hence the name. By optimising over the sum of cross-valiated log likelihoods, it is no longer strictly only assessing the log marginal likelihood, instead acting as more of a pseudo-likelihood. Directly optimising over the marginal likelihood provides the probability of observed data \textit{given model assumptions}, whereas the cross-validation approach provides the log predictive probability estimates independent of the fufilment of said model assumptions. \citep{wahba90} states in Chapter 4 that Gaussian cross validation methods are more robust to misspecification, including problems such as non-Gaussian errors. Cross-validation was chosen for this reason, as the intrinsic properties of the data were not studied in detail prior to performing experiments as a part of this study, and nor were biological experts consulted for the duration of the study to provide advice on the quality and soundness of predictive maps.

The log probability of the data ommitting training case $i$ from the $n$ input data points is:

$$\log p(y_i|X, \mathbf{y_{-i}}, \theta) = -\frac{1}{2}\log\sigma^2_i - \frac{(y_i - \mu_i)^2}{2 \sigma^2_i} - \frac{1}{2}\log2\pi$$

for input data $X$, the $i-th$ response variable $y$, predictive mean $\mu$, and predictive variance $\sigma^2$. The total log probability across all $n$ subsets of the data of size $n-1$ is then:

$$ L_{LOO}(X, y, \theta) = \sum^n_{i=1} \log p(y_i, X, \mathbf{y_{-i}}, \mathbf{\theta})$$

The LOO-CV predictive mean and variance can then be derived, and represented in terms of covariance matrix $K$, as calculated using \Cref{eq:simplegpcov} and \Cref{eq:fullgpcov}.
$$\mu_i= y_i - \frac{[K^{-1}\mathbf{y}]_i}{[K^{-1}]_{ii}} \text{ and } \sigma_i^2 = \frac{1}{[K^{-1}]_{ii}}$$

To be able to optimise hyperparemeters over the feature dimensions more efficiently, the partial derivates over \textit{each} is also needed:
$$\frac{\partial{u_i}}{\partial{\theta_j}} = \frac{[Z_j \alpha]}{[K^{-1}]_{ii}} - \frac{\alpha_i[Z_j K^{-1}_{ii}]_{ii}}{[K^{-1}]^2_{ii}}$$
$$\frac{\partial{\sigma_i^2}}{\partial{\theta_j}} = \frac{[Z_jK^{-1}]_{ii}}{[K^{-1}]^2_{ii}}$$

$$ \text{where } \alpha = K^{-1}\mathbf{y} \text{ and } Z_j = K^{-1} \frac{\partial{K}}{\partial{\theta_j}} $$

With the means to perform Gaussian process regression by determining the hyperparameters of the kernel using leave-one-out cross-validation, the next step would be to apply this to Gaussian process classification.

\section{Gaussian Process Classification} \label{chapsec:gpc}

As benthic habitat mapping requires the prediction of discrete labels and not continuous values, Gaussian process regression is not directly applicable to the problem domain. Just as logistic regression can be formulated by post-processing of the results of a linear regressor, Gaussian process classification can be performed by using multiple Gaussian process regressors as its underlying model. The one-vs-all approach is taken in this case, requiring $k$ separate Gaussian process regressors.

For example in \Cref{fig:ova-example}, when fitting a regressor for the green \textit{class 1}, the labels at coordinates corresponding to label $1$ are set to $1$, and all other points to $0$. For \textit{blue} label $2$, the labels of all the blue points are set to $1$, and the rest $0$, and so on so forth - this is applicable to any number of classes $k$, where the complexity increases linearly for each class that exists in the data.

\begin{figure}
    \includegraphics[scale=0.5]{ova_example.pdf}
    \caption{Simple example of data being split up to perform one-vs-all classification. The coloured lines corresponding to the label colours indicate the separation of clusters when performing each round of one-vs-all regression rounds. For class $3$, everything below the red line is taken to be the `positive label', and everything above, the `negative label'. This is only illustrative, and the labels will just as easily be wrangled if they were interspersed instead of segregated like they are here.}
    \label{fig:ova-example}
\end{figure}

When forming predictions, each of these separate Gaussian process regressors provide a different set of results as they consider separate one-vs-all cases as previously trained. For the $k$ labels, and each regressor $GP_k$, where $i=1,2...,k$:

\begin{equation}
    \mathbf{\bar{f_*}_{all}} = \begin{bmatrix}
        \mathbf{\bar{f_*}_{GP_1}} \\
        \mathbf{\bar{f_*}_{GP_2}} \\
        \ldots \\
        \mathbf{\bar{f_*}_{GP_{k-1}}} \\
        \mathbf{\bar{f_*}_{GP_k}} \\
    \end{bmatrix}, \mathbf{\mathbb{V}[f_*]_{all}} = \begin{bmatrix}
        \mathbf{\mathbb{V}[f_*]_{GP_1}} \\
        \mathbf{\mathbb{V}[f_*]_{GP_2}} \\
        \ldots \\
        \mathbf{\mathbb{V}[f_*]_{GP_{k-1}}} \\
        \mathbf{\mathbb{V}[f_*]_{GP_k}} \\
    \end{bmatrix}
\end{equation}

As the original labels were changed to be constrained in the range $[0, 1]$, the same is done to the predictive means, by passing it through the logistic sigmoid function (\Cref{eq:logistic}). The resulting matrix of predictive means and variances for any $k > 1$ provides a vector of $k$ predictions for each input rather than the single label that classification would require. To simplify the probabilistic results per label at each point, the value with the highest probability (i.e. the argmax) would be taken, along with the matching variance containing the confidence intervals. As the $O(n^3)$ operations required for the matrix inversions for performing Gaussian process regression must be performed $k$ times in total (once per label), the time required for anything more than several thousand points would make it impractical to use, requiring methods to sufficiently bring down the running time.

% \section{Subsampling for Gaussian Process experiments}
% \todo{(may need to take this section out - GPy wrapper deals with thousands of points no problem, no longer need to subsample with only 4000 after downsampling the data)}
% 
% Due to the $O(n^3)$ complexity of training a Gaussian Process Classifier, using all $16502$ points was infeasible, so it was necessary to use only a subsample of the training data. As can be seen in the above histograms \todo{(reference the figure instead. may need to combine them into one)}, the distribution of classes in both the simplified and non-simplified versions was very uneven. As a result of this skew, randomly sampling the the training data to fit our GP classifier against resulted in worse results than samplying an equal \textit{number} of points for each class. To obtain a reasonably well-performing set of 1000 points (the number chosen to obtain a balance between performance and time required), 10-fold cross validation was performed on random subsets of this size. To obtain the 1000 datapoints, both stratified sampling, as well as obtaining ratios of labels matching those in the training set were used. After 200 runs, the set with the best performance was used for the remaining GP experiments.

\section{Gaussian Process Approximation} \label{chapsec:gp-approx}

To be able to use Gaussian processes on larger data sets, approximations can be made to avoid paying the full cost of performing expensive operations, approximations can be made. The method of approximation that will be used in this study is ensemble methods. In the context of Gaussian processes, this involves breaking up the data into small chunks, where each `expert' performs regression seperately, before taking the product of all the experts' results to give the approximation for the full dataset. The property that these ensemble methods have in common are that any given expert's decisions are weighted by their precision (inverse of variance - so a low variance would result in a high precision), such that experts that are more `confident' in their results provide great input to the final weightings, whereas experts with low precision, or predictions that have a high variance, are unsure of their results, and hence provide comparably less input to the summarised predictions. The component that differs between the two approximations below (and these ensemble methods in general) is how the precision of each expert is used to weight their predictions.

\subsection{Product of Gaussian Process Experts}
The product of experts algorithm is a the simplest of Gaussian process ensemble methods, as the predictive variance of expert is directly used as-is to weight that particular expert's predictions, as shown in the equations below. The \textit{product} of experts described above is proportional to a single Gaussian process with the following product of expert (PoE) predictive mean and variance:

\begin{equation}
    \mu_*^{poe} = (\sigma_*^{poe})^2 \sum_k \sigma_k^{-2} (\mathbf{x_*}) \mu_k (\mathbf{x_*})
\end{equation}
\begin{equation}\label{eq:poe-precision}
    (\sigma_*^{poe})^{-2} = \sum_k \sigma_k^{-2} (\mathbf{x_*})
\end{equation}
where $\sigma_k^{-2}(\mathbf{x_*})$ are the predictive Gaussian precisions (inverse of Gaussian variances), and $\mu_k(\mathbf{x_*})$ are the predictive Gaussian means. \Cref{eq:poe-precision} describes the precision of the product of experts model in its entirety, by summing up the Gaussian precisions of all of the experts. For each point, this means that its PoE variance will be low if every expert calculates a low predictive variance, whereas if a sufficient number of them end up predicting a high variance for a given point, the overall PoE variance for that point will likely be high. This high variance is then applied to the PoE predictive means at every point, where the the product of each \textit{individual} expert's Gaussian precisions and Gaussian means are taken, and summed with all the other experts' at the same point, giving a higher precision ($\sigma_k^{-2}(\mathbf{x_*})$) a higher weight.

\begin{equation}
    \mu_*^{gpoe} = (\sigma_*^{gpoe})^2 \sum_k \beta_k \sigma_k^{-2} (\mathbf{x_*}) \mu_k (\mathbf{x_*})
\end{equation}
\begin{equation}
    (\sigma_*^{gpoe})^{-2} = \sum_k \beta_k \sigma_k^{-2} (\mathbf{x_*})
\end{equation}

The only difference in generalised product of Gaussian experts (GPoE) is the use of $\mathbf{\beta}$, where its purpose is to allow the explicit weighting of experts based on the circumstances. The value of each $\beta_k$ is flexible, but as scaling Gaussian processes to large datasets isn't the primary focus of this study, we simply set each $\beta_k$ to $\frac{1}{M}$, where $M$ is the number of experts, as suggested in \citep{deisenroth15} to be able to maintain reasonable margins of error. To show how they perform, below is a performance comparison between these two ensemble methods on a simple dataset:
\begin{figure}[H]
    \begin{minipage}{0.49\linewidth}
        \includegraphics[width=\linewidth]{gp_poe_with_variance_plot.pdf}
        \caption*{Product of Experts}
    \end{minipage}
    \hfill
    \begin{minipage}{0.49\linewidth}
        \includegraphics[width=\linewidth]{gp_gpoe_with_variance_plot.pdf}
        \caption*{Generalised Product of Experts}
    \end{minipage}
    \caption{\scriptsize{Examples of Gaussian process ensemble methods. As a result of the $\beta$ in the GPoE scaling the variances, they no longer cancel each other out when summed together, explaining the comparatively larger variance in the GPoE compared to the PoE, even when predictions are very close to the correct values. While these conservative (wide) variances can be a negative aspect as mentioned in \citet{deisenroth15}, it may also prove to be beneficial, as the experiments later show.}}
\end{figure}

% \subsection{Bayesian Committee Machine}
% \begin{equation}
%     \mu_*^{bcm} = (\sigma_*^{bcm})^2 \sum_{k=1}^M \sigma_k^{-2} (\mathbf{x_*}) \mu_k (\mathbf{x_*})
% \end{equation}
% \begin{equation}
%     (\sigma_*^{bcm})^{-2} = \sum_{k=1}^M \sigma_k^{-2} (\mathbf{x_*}) + (1-M)\sigma_{**}^{-2}
% \end{equation}
% where $\sigma_{**}^{-2}$ is the prior precision of $p(f_*)$, which itself is the inverse of the prior variances.
% 
% \subsection{Robust Bayesian Committee Machine}
% \begin{equation}
%     \mu_*^{rbcm} = (\sigma_*^{rbcm})^2 \sum_k \beta_k \sigma_k^{-2} (\mathbf{x_*}) \mu_k (\mathbf{x_*})
% \end{equation}
\section{Summary}

In this section, we explored the probabilistic capabilities of Gaussian processes regression, and how this translates into Gaussian process classification. As these presented very restrictive limits on the size of data that can be worked with, methods to estimate them by breaking down the algorithm into emabarrasingly parallelisable chunks were explored, along with how they compare in terms of performance. It would amiss at this point not to address the existence of multi-output Gaussian processes that can work with vectors of outputs for a given input datapoint~\citep{alvarez09}. However, the multi-output that they deal with is over arbitrary ranges. Because in this particular domain, normalised predictions per point must sum to $1$ (or the original total label count at a point, in the case of training data), multi-output Gaussian processes do not enforce this constraint, and as such, their use was not explored in this study. Without a way to correctly model the multi-output data using Gaussian processes, other methods that are able to need to be explored.


\input{mathbg/gp_large.tex}


%%%%% DM
\chapter{Dirichlet Multinomial Regression} \label{chap:dms}

Dirichlet multinomial regression, as the name suggestions, combines dirichlet and multinomial distributions to achieve the combined model. In particular, we are interested in modeling a distribution over category counts, as there exists relationship in our data such that every bathmetry point corresponds to a certain count of each possible label in the relevant area of benthos. \todo{explain why we should first revisit dirichlet, multinomial distributions separately before looking at dirichlet multinomial regression}

\section{Multinomial Distribution}
\todo {equations, description}

\section{Dirichlet Distribution}
\todo{descriptions}

$$\theta \sim Dir(\alpha) \text{ , dirichlet distributed random variable}$$ 
$$p(\theta)= \frac{1}{\beta(\alpha)} \Pi_{i=1}^n \theta_i^{\alpha_i - 1} I(\theta \in S) \text{ density function, I is indicator function}$$ 
$$ \theta = (\theta_1, ..., \theta_n), \alpha = (\alpha_1,...,\alpha_n), \alpha_i > 0 \text{ theta - n-dimensional vectors, alpha - parameters for distribution}$$
$$ S = \{x \in R^n : x_i \geq 0, \sum x_i = 1\} \text{ S is probability simplex, the set of pmfs on numbers 1 through n}$$ 
$$\frac{1}{\beta(\alpha)} = \frac{\Gamma(\alpha_0)}{\Gamma(\alpha_i) ... \Gamma(\alpha_n)}, \alpha_0 = \sum_{i=1}^n \alpha_i \text{ generalised beta function}$$

\section{Dirichlet Multinomial Regression}

\todo{descriptions}

$$DM(C|\alpha) = \frac{M!}{\Pi_k C_k!} \frac{\Gamma(\sum_k \alpha_k)}{\Gamma(\sum_k c_k + \alpha_k)} \Pi_{k=1}^k \frac{\Gamma(C_k + \alpha_k)}{\Gamma(\alpha_k)}$$
$$ M = \sum_k c_k $$

For the regressor, the two activation functions that were considered were exponential and softmax, where the former often provided better mapping predictions, but the latter is preferable in the general case due to its better numerical stability \todo {include graphs of exponential and softmax here}.
$$\alpha_k = \exp\{x^T w_k\}$$
$$\alpha_k = \text{softmax}\{x^T w_k\}$$

The weights $w$ here are in fact a matrix of weights with dimensions $(K \times D)$, where $K$ is the number of possible labels across the dataset, and $D$ is the dimensionality of the dataset. Muliplying the dirichlet multinomial prior by the likelihood then then gives the posterior over which to optimise to obtain the weights required to predict the normalised label counts at any given point.

This gives the joint-log-likelihood over both the dirichlet and multinomial distributions:
\begin{multline}
    \sum^N_{n=1} [\log(M_k) - \sum_k \log(c_k!) + \log \Gamma(\sum_k \alpha_k(x_n)) - \log \Gamma(\sum_k c_{nk} + \alpha_k(x_n))] \\
    + \sum^N_{n=1} \sum^K_{k=1} [\log \Gamma(c_k + \alpha(x)) - \log \Gamma(\alpha_k(x_n))] \\
    + \sum^K_{k=1} [-\frac{\phi}{2} \log(2\pi \phi) - \frac{1}{2}w_k^T \phi \mathbb{I} w_k]
\end{multline}

To optimise this equation, the partial derivative of the above over the weights $w$ are considered:
\begin{multline}
    \partial \frac{\log p(c, x)}{\partial w_k} = \sum_{n=1}^N x_n \alpha_k (x_n) [\psi(\sum_l \alpha_l(x_n)) - \psi(\sum_k c_{nk} + \alpha_k(x_n))] \\
    + \sum^N_{n=1} x_n \alpha_k (x_n) [\psi (c_{nk} + \alpha_k(x_n)) - \psi(\alpha_k(x_n))] - \frac{1}{\phi} w_k
\end{multline}

\todo {explain all the symbols here}

\subsection{Using MCMC instead of MAP}
\section{Illustrative Example}

The differences between a Gaussian Process that provides the probability distribution of possible labels compared to the Dirichlet Multinomial Regressor that provides the distribution of actual labels at a point, are highlighted in the illustrative example below. Note that three clusters were synthesised, with clusters A, B containing $0.7:0.3$ and $0.3:0.7$ average ratios in label mix per point respectively, while cluster C contained an even $0.5:0.5$ average split, where cluster had $100$ points. The colours on the overall plot are only representative of the \textbf{most} common label at each point - the actual distributions at each point are shown in the graphs following it.

% \todo{generate toy example with a mixed label A,B region and separate regions of mostly A, mostly B respectively}

\begin{figure}[H]
    \includegraphics[scale=0.7]{toydataplot.pdf}
    \caption{Plots of the three clusters, with labels taking on the argmax of each point}
    \label{fig:toyplot}
\end{figure}

\begin{figure}[H]
    \includegraphics[trim={0 2cm 0 2cm}]{toyplot_hist_distr_legend.pdf}
    \caption{Legend/axes for the following histogram plots showing distribution of labels at each point}
    \label{fig:toyplothist_legend}
\end{figure}

\begin{figure}[H]
    \begin{minipage}{.49\linewidth}
        \includegraphics[width=\linewidth]{toy_clusterA_distr.pdf}
        \caption{Label distribution of cluster A}
        \label{fig:toyclusterA}
    \end{minipage}
    \hfill
    \begin{minipage}{.49\linewidth}
        \includegraphics[width=\linewidth]{toy_clusterB_distr.pdf}
        \caption{Label distribution of cluster B}
        \label{fig:toyclusterB}
    \end{minipage}
    \begin{minipage}{.49\linewidth}
        \includegraphics[width=\linewidth]{toy_clusterC_distr.pdf}
        \caption{Label distribution of cluster C}
        \label{fig:toyclusterC}
    \end{minipage}
\end{figure}

In this example, the GP and DM models were each trained on half of each cluster, and made to predict the other half. However, as a standard GPC can only have single label inputs and outputs, a approximation/simplification was made for the purpose of calculating average error, whereby the label was simply taken to be the most frequently occurring label at any given point. While this is a reasonable simplification for clusters A, B as the dominant label has majority share, this is not the case for C, as the split between the two labels per point in the cluster is exactly even. In an initial attempt to counter this, multi-task GPs were considered as a means of making a \textit{fairer} comparison between a GP and DM, but the idea was ultimately discarded as it was not fit for purpose, one of the primary issues being that the model does not inherently restrict the outputs of a given datapoint to sum to $1$, instead being at the mercy of the parameters of the GP. 

\subsection{Results}

The results and plots for this example are below, and figures displayed were taken from an average of $20$ runs.

\begin{table}[H]
    \label{table:toy_gm_vs_gp}
    \begin{tabular}{|C{4cm}|C{4cm}|c|}
        \hline
        & Dirichlet Multinomial Regression RMSE* & Gaussian Process Classifier (argmax) RMSE \\\hline
        Original data & 0.070179271314358999 & 0.26833333333333337 \\\hline
        Quadratic-space projection & 0.065630111843395234 & 0.43433333333333335 \\\hline
        Cubic-space projection & 0.29019235800882354 & 0.43725490196076466 \\\hline
    \end{tabular}
    \begingroup
    \tiny{RMSE - root mean squared error}
    \endgroup
\end{table}

As can be seen from the above overvise, the DM performed best when projecting the data to quadratic space, while the GPC didbest on the original data as-is. This was taken into account for the plots below for the DM and GP respectively, which used an instance of the more favourably performing processed data. Note that the exact probabilities provided by the GP are hown in the following plots, in contrast to the argmax taken for error-calculation purposes. \todo{(don't use these graph plots here, do per-label heatmaps)}

\begin{figure}[H]
    \includegraphics[scale=0.6]{toy_dm_pred_plot0.pdf}
    \caption{DM Label distribution of label 0}
    \label{fig:toylabel0}
\end{figure}
\begin{figure}[H]
    \includegraphics[scale=0.6]{toy_dm_pred_plot1.pdf}
    \caption{DM Label distribution of label 1}
    \label{fig:toylabel1}
\end{figure} 
%\todo{dm - sitting on averages in each cluster. cause for concern?}

\begin{figure}[H]
    \includegraphics[scale=0.6]{gp_with_vars0.pdf}
    \caption{OvR GP performance with variance for label 0}
    \label{fig:toylabel0}
\end{figure}
\begin{figure}[H]
    \includegraphics[scale=0.6]{gp_with_vars1.pdf}
    \caption{OvR GP performance with variance for label 1}
    \label{fig:toylabel1}
\end{figure} 

As we can see, the DM performed notably better than the GP, with the predictions in each cluster staying true to the mean values. The GP, despite the $\sim 0.26$ error rate, can be seen to follow a consistent trend in the test data instead of adjusting to the two different classes. Most importantly, for both label 0, 1, the DM identifies that the portion of either label in cluster C is 0.5, due to its ability to learn and predict distributions at a point. As GPs are not designed to and cannot model this, it instead finds that neither label has a high probability, whilst also having a very high variance (beyond the $[0, 1]$ bound which again, the DM enforces but is not a property of a GP).

However, this is admittedly admittedly a rather simple example that assumes we have a sufficient amount of data from the \textit{three} possible habitat clusters - A by itself, B by itself, and a homogeneous mix of A and B, and as such, more detailed comparisons will be made using the full training dataset.

From this basic example, it is apparent that in the area where there is an even mix of labels A, B, the Gaussian Process' predictions are both noisy and very uncertain about their predictions, where human intervention would be required to observe the fact that it is in fact a consistent mix of both. In contrast, the dirichlet multinomial regressor is more confident in the fact that that area does in fact have a mix of labels.



