\chapter{Results} \label{chap:results}

The results from the Experiments detailed in Chapter 4 are listed below. The range of possible class values in some cases have been stretched beyond the existing class labels so that values align between different outputs to allow for easy, direct visual comparison. Note that the results to the above experiments will include those of both non-downsampled and downsampled results, as well as the full set of 24 labels as well as simplified ones.

Due to the low occurrence of some labels in the original dataset though, they have ended up being ommitted in predictions - these are excluded from the colour schemes of the benthic maps generated, so that those that do occur can be given more distinct colours from one another as to better differentiate between the habitats of a map, as well as allow a consistent comparison of across different maps.

In this section, the performance of common machine learning algorithms, namely kNN, Logistic Regression, Random Forest, and SVM are explored first, to provide a comparison to the later, more complex algorithms.

% The associated generated maps from the experiments are also provided here, but proper evaluation of them, such as what habitat clusters and relationships can be gleaned, will be explored in Chapter 5. 

\pagebreak
\section{Deterministic Methods}

\todo{results here currently outperform the argmax of GPs and DMs...hmm}

\begin{tabular}{|c|c|c|c|c|c|}
    \hline
    Algorithm & 10F-CV F1 & 10F-CV Accuracy & Parameters & Data \\\hline
        KNN         &  0.7180212553324016  &  0.8752880347998457 &   n = 5  & simple labels \\
Logistic Regression &  0.21221134330881872 &  0.7374253253308372 &            & simple labels \\
   Random Forest    &  0.8103659316744121  &  0.9117076519281244 &            & simple labels \\
        SVC         &  0.21221261743619468 &  0.7374255088743278 &     OvA    & simple labels \\
         DM         &   0.287405310254214  &  0.757925654489819  &            & simple labels \\
        KNN         &  0.4711662374785644  &  0.6510716003156948 &   n = 5  &  full labels  \\
Logistic Regression &  0.06457891531653175 & 0.25912130389295746 &            &  full labels  \\
   Random Forest    &  0.6075895500970125  &  0.7263355175008718 &            &  full labels  \\
        SVC         & 0.012413092166172946 & 0.16549499843988033 &     OvA    &  full labels  \\
         DM         &  0.13802716811804644 & 0.37856057852908254 &            &  full labels  \\
    \hline
\end{tabular}

\section{Gaussian Process Classification}

\todo{transfer all the results from markdown}

\todo{show more stratified results (not just even split) to show that even splits did better}

\begin{tabular}{|c|c|c|c|c|c|c|c|}
    \hline
    500     & Even       & GP     & 10F-CV & 10        & 0.86534    &                     &         \\
    500     & Stratified & GP     & 10F-CV & 10        & 0.80136    &                     &         \\
    1000    & Even       & GP     & All    & 1         & 0.87626    & Deterministic       & 0.56208 \\
    1000    & Even       & PoEGP  & All    & 5         & 0.80973    &                     & 0.47481 \\
    1000    & Even       & PoEGP  & All    & 200       & 0.80186    &                     & 0.47595 \\
    1000    & Even       & GPoEGP & All    & 5         & 0.80864    &                     & 0.51018 \\
    1000    & Even       & GPoEGP & All    & 200       & 0.80105    &                     & 0.47748 \\
    1000    & Even       & BCM    & All    & 5         & 0.80682    &                     & 0.48167 \\
    1000    & Even       & BCM    & All    & 200       & 0.80421    &                     & 0.48227 \\
    1000    & Even       & GPy    & All    & 1         & 0.87638    & RBF, EP (default)   & 0.57013 \\
    \hline
\end{tabular}

\todo{highlight areas with low/high certainty, etc.}

\section{Dirichlet Multinomial Regression}

\todo{transfer all the results from markdown}

\todo{highlight areas with biodiversity, etc.}
