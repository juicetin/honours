\pagebreak
\section{Biodiversity}

Another beneficial aspect of Dirichlet Multinomial Regression is that it inherently provides information about the distribution of different habitats in a given region, allowing observations on biodiversity to be made without extra steps such as clustering, which can be prohibitively expensive on datasets with millions of datapoints and tens (or more) of dimensions. Locating the co-existence of certain species would involve searching over the space of predictions for the desired distribution of habitats. 

As the 4-label case already aggregated similar classes from $24$ down to $4$, there was minimial biodiversity to observe over the query space, requiring us to perform predictions over the full $24$ labels to be able to find more abundant occurrences of biodiversity. To give a qualitative visual representation of the distribution of labels over the query space when using all $24$ labels, the predictive heatmaps for each of them is shown in \Cref{fig:dm24_0-7}, \Cref{fig:dm24_8-15}, and \Cref{fig:dm24_16-23}, where the numbers in brackets next to each title indicate the simplified label they correspond to. Because of the greater spread of labels, the majority of values are close to $0$, despite the possible range still being $[0,1]$ - this would result in a very uniformly purple plot if following the same colour scheme as previous plots. To allow for the subtle changes in predictions in the region close to $0$ to actually be visible, the log normal has been applied to all predictions so that the colours change rapidly for lower-valued predictions. Due to the minimumvalue threshold chosen, some points have been `ommitted' from the plots where there is white space - this indicates that their occurence rates were even lower than the lowest observable ones represented by dark purple.

\begin{figure}
    \includegraphics[clip, width=0.9\columnwidth, trim={0, 0.8cm, 0, 0.8cm}]{dm24_heatmaps_colourbar.pdf}
    \includegraphics[clip, width=\columnwidth]{dm24_heatmaps_0-3.pdf}\\
    \includegraphics[clip, width=\columnwidth]{dm24_heatmaps_4-7.pdf}
    \caption{Log-normalised predictions for labels $0-7$}
    \label{fig:dm24_0-7}
\end{figure}

\begin{figure}
    \includegraphics[clip, width=0.9\columnwidth, trim={0, 0.8cm, 0, 0.8cm}]{dm24_heatmaps_colourbar.pdf}
    \includegraphics[clip, width=\columnwidth]{dm24_heatmaps_8-11.pdf}\\
    \includegraphics[clip, width=\columnwidth]{dm24_heatmaps_12-15.pdf}
    \caption{Log-normalised predictions for labels $8-15$}
    \label{fig:dm24_8-15}
\end{figure}

\begin{figure}
    \includegraphics[clip, width=0.9\columnwidth, trim={0, 0.8cm, 0, 0.8cm}]{dm24_heatmaps_colourbar.pdf}
    \includegraphics[clip, width=\columnwidth]{dm24_heatmaps_16-19.pdf}\\
    \includegraphics[clip, width=\columnwidth]{dm24_heatmaps_20-23.pdf}
    \caption{Log-normalised predictions for labels $16-23$}
    \label{fig:dm24_16-23}
\end{figure}

Given the multi-label distribution predictions, there are any number of ways to either quantatively or qualitatively measure biodiversity. As an instance of the former, strict numerical conditions are used to define biodiversity for the example below. Depending on the aim of a particular environmental study or survey though, the definition of biodiversity is entirely flexible - it may refer to the co-existence of a specific few species or habitats ignoring all others, or it may be general biodiversity that defines cohabitation between any habitats. We take the latter of these formulations to show the power provided by the Dirichlet multinomial. For any given point (and by extension, region, where this pattern occurs frequently enough that the number of points fitting the criteria are dense enough to described as a cluster), $n$ number of labels are said to co-occur if for a given data point, the relevant area of benthos corresponding to a coordinate contains more than $n$ labels (number of `co-existing' habitats) that occur at a rate of at least $e$, where $0 < e \leq 1$. Given this specific `version' of biodiversity, $e$ would naturally need to decrease as $n$ increases - while it is possible to have large areas where at least two labels occur at a higher rate than $0.15$, this is of course not possible for seven points, for example ($7\times0.15=1.05$).

% \todo{[PLACEHOLDER]}
% We can visually observe that labels $7, 8, 10$ (variants of sand) occur in the same regions, with the following occupancy rates, beyond areas containing only negliglble traces ($\equiv0.00-0.02$) $7$ at a $\approx0.12$ occpuance rate, $8$ at $\approx0.18-0.20$, and $10$ at $\approx.17$.

\begin{figure}
    \makebox[\textwidth]{
        \includegraphics[scale=0.5]{dm24_cohab_map_2habs_143902points.pdf}
        \includegraphics[scale=0.5]{dm24_cohab_map_3habs_151814points.pdf}
    }
    \makebox[\textwidth]{
        \includegraphics[scale=0.5]{dm24_cohab_map_4habs_144647points.pdf}
        \includegraphics[scale=0.5]{dm24_cohab_map_5habs_134044points.pdf}
    }
    \makebox[\textwidth]{
        \includegraphics[scale=0.5]{dm24_cohab_map_6habs_123520points.pdf}
        \includegraphics[scale=0.5]{dm24_cohab_map_7habs_110622points.pdf}
    }
    \caption{Cohabitation for the stated occupancy threshold for the label of each plot, from $2-7$.}
\end{figure}

\begin{figure}
    \makebox[\textwidth]{
        \includegraphics[scale=0.5]{dm24_cohab_map_8habs_78805points.pdf}
        \includegraphics[scale=0.5]{dm24_cohab_map_9habs_57530points.pdf}
    }
    \makebox[\textwidth]{
        \includegraphics[scale=0.5]{dm24_cohab_map_10habs_42303points.pdf}
        \includegraphics[scale=0.5]{dm24_cohab_map_11habs_27102points.pdf}
    }
    \makebox[\textwidth]{
        \includegraphics[scale=0.5]{dm24_cohab_map_12habs_17774points.pdf}
        \includegraphics[scale=0.5]{dm24_cohab_map_13habs_9711points.pdf}
    }
    \caption{Cohabitation for the stated occupancy threshold for the label of each plot, from $8-13$.}
\end{figure}

For the definition of biodiversity taken in the above plots, it is easy to observe the biodiversity over the 24 labels in general, with a general trend that biodiveristy between an increasing number of habitats, even with the decreasing threshold, results in a lower density of biodiverse areas. The initial threshold when considering $2$ labels was $0.15$, dropping by $10\%$ (or $0.9$ of the previous value) for each additional label considered. For these particular parameters, any signs of visibly continuous areas of biodiversity disappear when considering more than $13$ labels.

This approach can be easily modified to target the tracking of specific habitats - for example, if data is collected periodically in any area where coral bleaching is suspected to occur, predictive maps can be generated at the same intervals, allowing the \textit{changes} in biodiversity to be observed. Although such a task can be more generally achieved even using deterministic methods by simply taking a count of labels over the query space to see positive/negative trends, this is quite coarse in comparison, and specifying with accuracy the regions where change occurred involves extra work - for example, to assess if an area that was previously unbleached coral had become bleached through automated means, an individual working with the data may have to define an algorithm that checks the area around points up to a certain distance, then observe any potential changes in predictions generated from newly collected data, requiring considerably more effort in designing a method that allows analysis to be done efficiently. In contrast - performing the checks for general biodiversity using the above biodiversity across all 24 labels only took $11$ seconds, with checks for individual labels take less than $1$ second.

% \includegraphics{dm24_cohab_map_14habs_6928points.pdf}
% \includegraphics{dm24_cohab_map_15habs_3452points.pdf}
% \includegraphics{dm24_cohab_map_16habs_1646points.pdf}
% \includegraphics{dm24_cohab_map_17habs_1561points.pdf}
% \includegraphics{dm24_cohab_map_18habs_1893points.pdf}
% \includegraphics{dm24_cohab_map_19habs_2208points.pdf}
% \includegraphics{dm24_cohab_map_20habs_1869points.pdf}
% \includegraphics{dm24_cohab_map_21habs_703points.pdf}
% \includegraphics{dm24_cohab_map_22habs_569points.pdf}
% \includegraphics{dm24_cohab_map_23habs_156points.pdf}
