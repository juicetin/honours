\chapter*{Abstract}

Being able to predict the state of benthic habitats based on limited information is crucial for environmental conservation, particularly as the impact of human activity on our oceans is greater than ever before. A considerable portion of work done in the area uses deterministic methods that strictly assign only one label to a given bathymetry data point, while more advanced models provide probabilistic results over all possible labels at any one point, also similarly only representing a single output. However, like the majority of real life classification problems \todo{(citation here perhaps)}, habitat mapping is intrinsically a multi-label problem for any data collected at a resolution low enough to be economically feasible to be performed at a large scale. In this paper, we explore advantages of having probabilistic class outputs as well as treating benthic habitat mapping as a multi-output problem particularly when faced with relatively low resolution bathymetry data, compared to the primary method of deterministic, single-output methods explored in existing literature.
