\chapter{Introduction}

Earth's oceans cover 70\% of its surface, but only less than 10\% of the Earth's oceans have been explored to date\footnote{Oceanservice.noaa.gov. (2016). How much of the ocean have we explored?. [online] Available at: http://oceanservice.noaa.gov/facts/exploration.html}. There have been increasing efforts over the past few decades to more efficiently map out these unexplored areas to monitor marine ecosystems to be able to track the state of them over time for management, preseveration, etc. purposes. The process used is called benthic habitat mapping, which is the process of generating predictive maps of different habitat types at the bottom of a body of water. Most studies looking to create benthic habitat maps share some basic key steps - acoustic data is used to estimate properties about the surface of the water, which are then mapped to, using machine learning algorithms, \textit{in situ} data such as still images, videos, or samples of the area in question. It is the relationship which is inferred between the different data sets inferred using machine learning techniques that varies between studies. A considerable portion of such studies are shown to use deterministic methods to predict a label for any given coordinate such as Random Forests and Support Vector machines (SVMs), whilst more recent ones make use of more informative methods such as Gaussian Processes, providing a distribution over all possible labels given any data point.

\section{Contribution}

The main contribution of this thesis will be to explore how to use data where a single data point does not only have one label exclusively, but instead corresponds to a tally of each possible label. For example, a particular 5m x 5m area in the benthos may be an even mix of both sand and coral, but in previous literature, the data was simplified such that whichever label occurred more frequently regardless of how small the margin would be the single label assigned to that point. This results in a very coarse approximation even when using Gaussian Processes attempts to model the uncertainty/uncertainty with its predictions at each point (but ultimately only provides a single, final prediction). To alleviate this and provide a richer set of information, we explore the use of Dirichlet Multinomials, which provides a distribution of each label that represents something entirely different. Whereas in a Gaussian Process, each label is assigned the probability of being the correct one, the output of a Dirichlet Multinomial Regressor provides the distribution of the frequency of labels in a particular space itself. See section \todo{GP vs DM} for an illustrative example on how results would differ in practice between the two methods.

\section{Motivation}

The motivation behind assessing the effectiveness and advantages of such a method are that they inherently tie in with lower resolution data, particularly when a single images corresponds to a large enough area such that one would expect a mix of different labels. This is advantageous because we want to be able to re-sample data from any given site periodically (for example, every 3-4 years) whilst being economically efficient. This naturally lends to lower resolution data, meaning that summarising large areas to a single label would theoretically be throwing away a majority of the information contained in bathymetry and image data.

\section{Outline}

We will first look at the existing literature in \autoref{chap:litreview} - on collection of bathymetry and image data briefly, then on deterministic approaches to benthic habitat mapping to date, such as logistic regression, and random forests, and their performance on varying types of benthic environments. This is contrast the more informative probabilistic and multi-output approaches that will be explained in \autoref{chap:gps} and \autoref{chap:dms}, where we look at the mathematical background behind Gaussian Processes and Dirichlet Multinomial Regression. In \autoref{chap:experiments}, we then apply the techniques explained in the previous chapters and observe their performance, points of interest, as well as how the information obtained differs to methods visited in \autoref{chap:litreview}. 
