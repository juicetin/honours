\documentclass[10pt,sts]{article}
\usepackage[margin=0.7in]{geometry}
\usepackage{multicol}               % Columns
\usepackage{tabularx}               % Extra table functionality
\usepackage{amssymb}
\usepackage{amsmath}
\setlength{\skip\footins}{1.0cm}    % More space between text and footnotes
\usepackage{color}
\usepackage{picinpar}
\usepackage{graphicx}
\graphicspath{ {/home/justinting/programming/jt-honours/images/} }
\usepackage{wrapfig}
\usepackage[export]{adjustbox}
\usepackage{xcolor}
\usepackage[hyperfootnotes=false]{hyperref}
\hypersetup{
    colorlinks,
    citecolor=blue,
    linkcolor=red,
    urlcolor=blue,
    citebordercolor=red,
    filebordercolor=red,
    linkbordercolor=blue
}
\usepackage[round]{natbib}          % Round brackets when citep*-ing sources
\usepackage{comment}                % Comments...
\usepackage{titlesec}
% \USEpackage{pgfplots} % Don't have this atm
\usepackage[parfill]{parskip}       % Newline instead of indentation per paragraph
\usepackage{enumitem}
\setlist{nosep}
\setlength{\columnsep}{0.7cm}       % Separate columns (when used) by 0.5cm
% \setcounter{secnumdepth}{4}         % Give paragraphs 'numbers'
\newcommand\gauss[2]{1/(#2*sqrt(2*pi))*exp(-((x-#1)^2)/(2*#2^2))}

\title{Large Scale Gaussian Processes in Benthic Habitat Mapping}
\author{Justin Ting, 430203826}
\date{October 2016}

\begin{document}
\maketitle

    \section{Abstract}
    Being able to predict the state of benthic habitats based on limited information is crucial for environmental conservation, particularly as the impact of human activity on our oceans is greater than ever before. A large portion of work done in the area uses deterministic methods that strictly assign only one label to a given bathymetry data point, while more advanced models provide probabilistic results over all possible labels at any one point. However, like the majority of real life classification problems, this one is intrinsically a multi-label problem for any data collected at a resolution low enough to be economically feasible to be performed at a large scale. With the motivation of working with (relatively) low resolution bathymetry data, we explore the advantages of treating benthic habitat mapping as a multi-label problem compared to combinations of deterministic, probabilistic, and single-label methods used in previous works.

    \section{Introduction}
    * need for probabilistic outputs from benthic habitat mapping predictions
    * for image-based, a particular location could include multiple labels (habitat classes)
    * real life problems are usually multi-task - but for the majority of machine learning applications, they are converted into simplified single-task problems

    \subsection{Overview}

    \section{Literature Review}

    \section{Data}

    \section{Experiments}

    \section{Results}

    \section{Evaluation and Discussion}

    \section{Conclusion}

    \section{Future Work}

    Dummy citation: ~\citep{halpern08}

\bibliographystyle{plainnat}

\bibliography{Bibliography}

\end{document}
