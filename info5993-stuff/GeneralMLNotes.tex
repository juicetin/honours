\documentclass[12pt]{article}
\usepackage[margin=0.7in]{geometry}
\usepackage{multicol}               % Columns
\usepackage{tabularx}               % Extra table functionality
\setlength{\skip\footins}{1.0cm}    % More space between text and footnotes
\usepackage[round]{natbib}          % Round brackets when citep*-ing sources
\usepackage{comment}                % Comments...
\usepackage{titlesec}
\usepackage[parfill]{parskip}       % Newline instead of indentation per paragraph
\usepackage{enumitem}
\setlist{nosep}
\setlength{\columnsep}{0.5cm}       % Separate columns (when used) by 0.5cm
\setcounter{secnumdepth}{4}         % Give paragraphs 'numbers'

\begin{document}
\begin{itemize}
    \item simple linear regression is one quantitative variable predicting another 
    \item multiple regression is simple linear regression with more independent variables 
    \item nonlinear regression is still two quantiative variables, but data is curvilinear 
    \item binary data does not have a normal distribution - a condition needed for most other types of regression 
    \item predicted variables of the dependent variable can be beyond 0/1 in other types of regressions - logistical regression deals with probabilities 
    \item   + probabilities have to be between 0 and 1 
    \item probabilities are often not as linear such as U shapes where probability is very low/high at extreme x-values
    \item   + e.g. probability (may be, as an example) higher for contracting the flu for oldies/young'uns, but lower for middle aged people
    \item odds of event X - P(X)/(1-P(X))
    \item the odds ratio for a variable in logistic regression represents how the odds change with a 1 unit increase in that variable holding all other variables constant
    \item odds vs probability
    \item   + underlying probability may be low, but odds could increase in magnitude very quickly
    \item   + e.g. probabilities of being struck by lightning, and being struck by a meteor
    \item       * probability of being hit by lightning are *much much* higher than being struck than a meteor
    \item       * however, being hit by lightning is a very low prob to begin with
    \item in logistic regresion, we are estimating an unknown p for any given lienar combination of the independent variables
    \item link together independent variables to essentially the Bernoulli distribution; that link is called the logit
    \item in logistical regression, we don't know p like we do in Binomial (Bernoulli) distribution problems - goal of logistical regression is to estimate p for a linear combination of the independent variables
    \item the natural log of the odds ratio, the logit, is that link function
    \item $ln(odds)$ $==>$ $ln(\frac{p}{(1-p)})$ is the $logit(p)$ OR $ln(p) - ln(1-p) = logit(p)$
    \item reminder: $log_{e}x = lnx$
    \item boundaries - if p = 1 or 0, our function is undefined, when p = 0.5, our function = 0 (our function being the logit)
    \item logit function graphs to a sigmoid function
    \item 0-1 ran along x-axis but we want probabilities to be on the y-axis - achieve this by taking inverse of the logit function
    \item $logit^{-1}(\alpha) = \frac{1}{1+e^{-\alpha}} = \frac{e^{\alpha}}{1+e^{\alpha}}$ where $\alpha$ = some number
    \item   + $\alpha$ is the linar combination of variables and their coefficients - inverse-logit will return the probabilyt of being a "1" or in the "event occurs" group
    \item inverse logit A.K.A mean function
    \item   + $\mu_{y|x}$
    \item note about coefficients
    \item   + coeffs calculated using maximum likelihood estimation
    \item estimated regression equation
        \begin{itemize}
            \item natural log of odds ratio is equivalent to linear function of the independent variables - antilog of logit function allows ut to find the estimated regression equation, i.e. solve for p, the estimated probabibility
            \item $logit(p) = ln()= \beta_{0} + \beta_{1}x_{1}$
            \item isolate/solve $p$! - going to do this using algebra
            \item $\frac{p}{1-p} = e^{\beta_{0} + \beta_{1}x_{1}}$ // on the right here is Euler's constant raised to the power of a linear combinatino of all the independent variables
            \item $p = e^{\beta_{0}+\beta_{1}x_{1}}(1-p)$
            \item $p = e^{\beta_{0}+\beta_{1}x_{1}} - e^{\beta_{0}+\beta_{1}x_{1}}p $
            \item $p + e^{\beta_{0}+\beta_{1}x_{1}}p = e^{\beta_{0}+\beta_{1}x_{1}}$
            \item $p(1+e^{\beta_{0}+\beta_{1}x_{1}}) = e^{\beta_{0}+\beta_{1}x_{1}}$
            \item estimated regression equation! $\hat{p} = \frac{e^{\beta_{0}+\beta_{1}x_{1}}}{1+e^{\beta_{0}+\beta_{1}x_{1}}}$
        \end{itemize}
    \item watched up to Brandon Foltz' Statistics 101: Logistic Regression, Odds Ratio for Any Interval
\end{itemize}

\section{Random Forests}
\begin{itemize}
    \item combination of learning models increases classification accuracy (bagging)
    \item bagging - average noisy and unbiased models to create a new model with low variance
    \item random forest algorithm works as a large collection of decorrelated decision trees \footnote{Gupta, Ashish. Learning Apache Mahout Classification. Packt Publishing Ltd, 2015.}
    \item create random subsets of the original full set of values - e.g. out of 100, one particular subset might contain all features of rows 1, 5, 9, 24, 38, 49, 52, 68, 72, 82 and their classifications, others follow
    \item create decision tree from each subset
    \item 'forest' - lots of decision trees
    \item count votes for each class when an unknown class feature set is passed through all the decision trees - take majority for classification, average for regression
    \item random forests as a supervised learning algorithm came only second to boosted decision trees in a comparison study \footnote{Caruana, Rich, and Alexandru Niculescu-Mizil. "An empirical comparison of supervised learning algorithms." Proceedings of the 23rd international conference on Machine learning. ACM, 2006. APA } (third was bagged decision trees, SVM 4th - essentially, aggregated tree methods were at the top)
    \item at each node in tree, choose random subset of (m - constant in trees) features - consider only splits on those features
    \item each individual tree has high variance - but by averaging over an ensemble of trees, we reduce the variance in the final estimator, and hence lowering error
    \item introduced by Breiman
\end{itemize}

\section{Multivariate Gaussian}
\begin{itemize}
    \item univariate Gaussian - $X\\~N|\mu,\sigma^{2}$
    \item means X has density $f(x) = <..insert f here...>$
    \item degenerate univariate Gaussian - $\sigma^{2}$ can be 0, i.e. $X == \mu$
    \item multivariate Gaussian - A random variable $X \in R^{n}$ is multivariate Gaussian (or normal or multivariate normal) if any linear combination of its components is univariate Gaussian i.e. $a^{T}X = \Sigma a_{i}X_{i}$ is Gaussian for every $a \in R^{n}$
    \item defn: $X~N(\mu, C)$ (C is a pos semidef matrix) means X is Gaussian with $EX_{i}=\mu_{i}$ and $covariance(X_{i},X_{j}) = C_{ij}$
    \item $\mu and C$ uniquely determine the distribution $N(\mu, C)$
    \item positive semidef matrix - all eigenvalues are >= 0
\end{itemize}

\section{Gaussian Processes}
\begin{itemize}
    \item most often used in geostatistics - modeling spatial things - geography, meteorology, etc.
    \item things evolving over time, where its going, e.g. an airplane
    \item finance, physics, diffusion processes, all the things!
    \item Gaussian Processes - for any set $S$, a Gaussian process (GP) on $S$ is a set of random variables ($Z_{t} : t\in S$)
    \item for any N, and for any $t_{1}...t_{n}\in S$, ($Z_{t_{1}},...,Z_{t_{n}}$) is multivariate Gaussian
    \item finite dimentional distributions - distribution of all the t in Z vectors
    \item Example: $S = {1,...,d}$      $(Z_{t}) = (Z_{1},...,Z_{d}) \in R^{d}$ - satisfies definition, trivial
    \item random lines: $S=R$, $Z_{t} = tW, W~N(0,1)$
    \item 
\end{itemize}

\end{document}
