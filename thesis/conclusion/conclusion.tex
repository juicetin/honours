\chapter{Conclusion} \label{chap:conclusion}

Benthic habitat mapping is a relatively old concept, dating back at at least several decades when photos and videos became a viable method of capturing information about the benthos ~\citep{gibson07} as an alternative to earlier destructive methods of sediment sampling. However, the different sources of data required and the machine learning techniques needed to model the data to be able to predict properties about the benthos did not become more readily available until relatively recently. The ability to collect data more easily then before using tools such as multibeam echosounders and autonomous underwater vehicles has seen an increase in numbers of studies assessing predictive ability of different types of \textit{data}, using popular machine learning algorithms such as SVMs and random forests. More complex, state of the art models such as Gaussian processes are able to provide probabilistic outputs ~\citep{bender12} along with error margins for these probabilties.

In this study, products of experts models were applied to Gaussian processes to lift the usual limitation on data sizes of several thousand points due to the time that would be needed to fit them. In fact, due to the requirement of multiple Gaussians processes to form GP classifiers, this several thousand is effectively scaled down by a factor proportional to the number of classes in the data. Experiments showed that for aggregated labels, the ensembles of experts were comparable in performance to the full GP, though the performance gap grew when dealing with the $24$-label case, keeping in mind that the GP became unusable on the full query data for predictions due to the time required. The predictions on the simple labels of the approximations also matched that of random forests and the distributions of the Diriclet multinomial's predictions, with the latter being able to utilise the data in its original, rich form.

As studies performed in the area of benthic habitat mapping deal with single-label outputs, they would be unable to account for and fully utilise multi-output label counts, and resort to approximations by taking the most frequently occuring label per output. To assess the viability of working with multi-output data in benthic habitat mapping, the Dirichlet multinomial was used, serving exactly this purpose. This allowed richer information to be obtained from predictions even compared to GPs, by modeling the distributions of labels at every point, and not be forced to make approximations such as taking the most probable or frequent label to be able to represent data in a valid form. A DM's entropy allowed quantification of its certainty in predictions, similar to a GP's predictive probabilities over labels and corresponding variances. This then allowed observations on habitat densities that previously were not possible with single-label outputs, as well as ecological properties such as biodiversity that otherwise require additional processing of single-output predictions.

This study has shown the viability of using \textit{all} image data collected when accompanied by bathymetry data, and not needing to only take the nearest image so that standard algorithms such as SVMs and random forests can be used, allowing the  utilisation of all data and observations of habitat occurrence rates and biodiversity to be made. Moreover, whereas even GP approximations took over an hour to perform parallelised predictions for the simplified labels, the DM only required seconds to train and predict when using the full $24$ labels.

\section{Future Work}

As the predictions using a Dirichlet multionimial allow for easy observations of habitat occurence rates as well as biodiversity, it could potentially serve as a superior tool for monitoring changes in benthic habitats over time. To assess its effectiveness in such a context, periodic data from the same location would be needed, and predictive maps generated for each period's available data. Since the data being used is historical, existing measurements of changes in particular habitats confirmed using other methods can be used to verify the correctness of the Dirichlet multinomial. As the collection of bathymetry data and images may be more costly than other methods designed to measure one specific point of interest, it would be more practical when seeking to understand a particular benthic area as a whole. To be able to capitalise on the predictive maps being generated and the implications of any detected changes, it would also be ideal to have experts in marine biology and marine ecosystems to provide input during experimental phases, as opposed to relying on general predictive power (e.g. accuracy) alone.

Another aspect of this study worth further investigation is the determining of a Dirichlet multinomial's $\alpha$ parameters using the softmax activation function in \Cref{eq:softmax}. This represents a basic, \textit{derministic} relationship between the between the data and the Dirichlet's weight parameters, whereas using another model such as a Gaussian as the underlying function may provide more informative, and possibly better results. Given that previous works have explored the use of Gaussian process classification on marine vessels to maximise the effiency of data-collecting machines by focusing on `uncertain' areas ~\citep{rigby10}, it would be interesting to pursue a similar strategy for the Dirichlet multinomial that may be inherently better suited to interepreting all the data of its surroundings (where the immediate area surrounding a vehicle would be a mix of habitats). If the underlying activation function were a Gaussian as well, this would provide an extra layer of probabilistic certainties (or otherwise) in predictions to allow better-informed decisions to be made, autonomous (for data collection) or otherwise (for benthic habitat management/conservation).
